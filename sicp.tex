\documentclass[twoside]{book}%Опция draft помечает overfull'ы. Выкинем.
\usepackage{etex}
\usepackage[utf8]{inputenc}
\usepackage[english,polutonikogreek,russian]{babel}	
\usepackage{a4wide}
\usepackage{russcorr}
\usepackage{pictex}
\usepackage{fancyvrb}
\usepackage{epigraph}
\usepackage{multind}
\usepackage{graphics}
\usepackage{synttree}
\usepackage{pscyr}
\usepackage{indentfirst}

\usepackage[colorlinks,hyperindex=false]{hyperref}
\hypersetup{
    pdfinfo={
        Title={Структура и интерпретация компьютерных программ},
        Author={Харольд Абельсон, Джеральд Джей Сассман}
    }
}

%\usepackage[mirror]{crop}
%\usepackage{epic}
%\usepackage{eepic}
% Нумеровать подподсекции
\setcounter{secnumdepth}{3}

\makeatletter

\makeatother

% Отменить линии под эпиграфами
\setlength{\epigraphrule}{0pt}
%\renewcommand{\epigraphsize}{\small}
% разрешить команды и математику в листингах
\fvset{commandchars=\\\{\},codes={\catcode`$=3\catcode`^=7\catcode`_=8}}


% разрешить листинги в сносках
\VerbatimFootnotes


\makeindex{ru}
\makeindex{en}

\makeatletter

\usepackage{sicp}

\pagestyle{myheadings}
%%%%%%%%%%%%%%%%%%%%%%%%%%%%%%%%%
%%%%%% END OF THE PREAMBLE %%%%%%
%%%%%%%%%%%%%%%%%%%%%%%%%%%%%%%%%

%%%%%%%%%%%%%%%%%%%%%%%%%%%%%%%%%
%%%%%%%% THE BOOK BEGINS %%%%%%%%
%%%%%%%%%%%%%%%%%%%%%%%%%%%%%%%%%
\begin{document}
%\pagestyle{empty}
\renewcommand{\@makechapterhead}[1]{
  \vspace*{10pt}%
  {\parindent=0pt
   \hrule
   \par\medskip
   \centering \huge\sc Глава{}
   \thechapter
   \par
   \vspace{10pt}
   \huge {\bf\sc{#1}}
   \par\medskip
   \hrule
   \nopagebreak
   \vspace{10pt}
}}

\hfuzz=0,3pt
\begin{titlepage}

\begin{center}
{\sf\huge
Harold Abelson\\[20pt]

{\large and} Gerald Jay Sussman\\[20pt]

{\large with} Julie Sussman \\[150pt]

Structure and Interpretation \\[6pt]
of Computer Programs}

\vfill

{\sf\normalsize 
The MIT Press\\
Cambridge, Massatchusetts \hfill London, England

The McGraw-Hill Companies, Inc.\\
New York \hfill St.Louis \hfill San Francisco \hfill Montreal \hfill Toronto 
}

\clearpage
\thispagestyle{empty}
{\sf\huge
Харольд Абельсон \\[20pt]

Джеральд Джей Сассман\\[20pt]

{\large при участии} Джули Сассман\\[150pt]

Структура и интерпретация \\[6pt]

компьютерных программ
}
%%
%\vfill

{\sf
Добросвет, 2006
}

\end{center}
\clearpage
\mbox{}
\clearpage
\thispagestyle{empty}
\begin{quote}
%%%Кончились титульные листы
Эта книга посвящается, с уважением и любовью, духу,
который живет внутри компьютера.\\[10pt] %}{}\epigraph{

``Мне кажется, чрезвычайно важно, чтобы мы, занимаясь
информатикой, получали радость от общения с компьютером.  С самого
начала это было громадным удовольствием. Конечно,
время от времени встревали заказчики, и через какое-то время мы стали
серьезно относиться к их жалобам. Нам стало казаться, что мы
вправду отвечаем за то, чтобы эти машины использовались успешно и безошибочно.
Я не думаю, что это так. Я считаю, что мы
отвечаем за то, чтобы их тренировать, указывать им
новые направления и поддерживать уют в доме. Я надеюсь, что
информатика никогда не перестанет быть радостью. Я надеюсь, что
мы не превратимся в миссионеров. Не надо чувствовать себя
продавцом Библий.  Таких в мире и так достаточно. То, что
Вы знаете о программировании, могут выучить и другие. Не думайте, что 
в ваших руках ключ к успешной работе с компьютерами. Что у Вас, как я
думаю и надеюсь, есть~--- это разум: способность увидеть в машине
больше, чем Вы видели, когда Вас впервые к ней подвели, увидеть, что
Вы способны сделать ее б\'oльшим.''%{

Алан Дж. Перлис (1 апреля 1922 -- 7 февраля 1990)
\end{quote}
\end{titlepage}
\pagestyle{headings}
%\setcounter{page}{5}
\tableofcontents
\thispagestyle{empty}
%\sloppy

\newpage
\pagestyle{headings}
\input preface-1.tex
%\clearpage

\input preface-2.tex
%\clearpage

\input preface-3.tex
%\clearpage

\input preface-4.tex
\clearpage
%\pagenumbering{arabic}

\pagestyle{headings}
%% Закрываем группу, в которой действует нестандартное 
%% оформление колонтитулов, потому что в главах все и так нормально делается
\input chapter-1.tex

\input chapter-2.tex

\input chapter-3.tex

\input chapter-4.tex

\input chapter-5.tex


\clearpage
\pagestyle{myheadings}
%%Опять делаем нестандартные колонтитулы для литературы и индекса
\input references.tex
\clearpage

\chapter*{Предметный указатель}
\markboth{Предметный указатель}{Предметный указатель}
\addcontentsline{toc}{chapter}{Предметный указатель}
\epigraph{
Все неточности в этом указателе объясняются тем, что его готовили при
помощи вычислительной машины.
}{
Дональд~Э. Кнут, {\it Основные алгоритмы} ({\it Исскусство
  программирования для ЭВМ}, том 1)
}
\index{ru}{Кнут, Дональд~Э.||Donald~E. Knuth||n|}

\noindent
Номера страниц для определений процедур даны курсивом.\\
Буква {\it п} после номера страницы отсылает к примечанию.\\
Буквы {\it нс} после названия элементарной функции либо особой формы
означают, что она не входит в стандарт Scheme IEEE.\\
\begin{raggedright}
\begin{multicols}{2}
\sloppy
\input ru-idx.tex
\end{multicols}
\end{raggedright}
\end{document}

Alyssa P. Hacker - Лиз П. Хакер
Eva Lu Ator - Инта Претатор
Ben Bittidle - Бен Битобор
Louis Reasoner - Нед О'Тумкал
Lem E. Tweakit - Дайко Поправич
Cy D. Fect - Сай Д. Фект
Con Fiden - 
Aull DeWitt 
Oliver Warbuck

