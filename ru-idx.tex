\begin{theindex}
\item {\texttt{"} (двойная кавычка)} 147{\it п}
\item {\texttt{`} (обратная кавычка)} 524{\it п}
\item {\texttt{'} (одинарная кавычка)} 147{\it п}
  \subitem {и \texttt{read}} 356{\it п}, 444{\it п}
\item {\texttt{<} (элементарный предикат сравнения чисел)} {\it 36}
\item {\texttt{!}, в именах процедур} 214{\it п}
\item {\texttt{*} (элементарная процедура умножения)} {\it 26}
\item {\texttt{+} (элементарная процедура сложения)} {\it 26}
\item {\texttt{,} (запятая, внутри обратной кавычки)} 524{\it п}
\item {\texttt{-} (элементарная процедура вычитания)} {\it 26}
  \subitem {как смена знака} {\it 36}{\it п}
\item {\texttt{/} (элементарная процедура деления)} {\it 26}
\item {;} {\it см.} точка с запятой
\item {\texttt{=} (элементарный предикат сравнения чисел)} {\it 36}
\item {\texttt{=number?}} {\it 153}
\item {\texttt{=zero?} (обобщенная)} 191 (упр.~2.80)
  \subitem {для многочленов} 204 (упр.~2.87)
\item {\texttt{>} (элементарный предикат сравнения чисел)} {\it 36}
\item {\texttt{>=} (элементарный предикат сравнения чисел)} {\it 38}
\item {\texttt{?},~в именах предикатов} 41{\it п}
\item {$\mapsto$, математическая запись для функций} 81{\it п}
\item {$\lambda$-исчисление} {\it см.} лямбда-исчисление
\item {$\pi$} {\it см.} пи
\item {$\sum$} {\it см.} сигма-запись
\item {$\Theta(f(n))$} {\it см.} тета от $f(n)$
\bigskip
\item {\texttt{abs}} {\it 36}, {\it 37}
\item {\texttt{accelerated-sequence}} {\it 316}
\item {\texttt{accumulate}} 74 (упр.~1.32), {\it 123}
  \subitem {то же, что \texttt{fold-right}} 127 (упр.~2.38)
\item {\texttt{accumulate-n}} 126 (упр.~2.36)
\item {\texttt{actual-value}} {\it 373}
\item {Ada (Ада)}
  \subitem {рекурсивные процедуры} 51
\item {\texttt{add} (обобщенная)} {\it 187}
  \subitem {примененная к коэффициентам многочленов} 202
\item {\texttt{add-action!}} 264, {\it 266}
\item {\texttt{add-binding-to-frame!}} {\it 352}
\item {\texttt{add-complex}} {\it 173}
\item {\texttt{add-complex-to-schemenum}} {\it 191}
\item {\texttt{add-interval}} {\it 103}
\item {\texttt{add-lists}} {\it 380}
\item {\texttt{add-poly}} {\it 200}
\item {\texttt{add-rat}} {\it 94}
\item {\texttt{add-rule-or-assertion!}} {\it 441}
\item {\texttt{add-streams}} {\it 309}
\item {\texttt{add-terms}} {\it 201}
\item {\texttt{add-to-agenda!}} 267, {\it 270}
\item {\texttt{add-vect}} 141 (упр.~2.46)
\item {\texttt{addend}} 150, {\it 152}
\item {\texttt{adder} (элементарное ограничение)} {\it 275}
\item {\texttt{adjoin-arg}} {\it 504}{\it п}
\item {\texttt{adjoin-set}} 154
  \subitem {для множеств взвешенных элементов} {\it 168}
  \subitem {представление в виде бинарных деревьев} {\it 159}
  \subitem {представление в виде неупорядоченных списков} {\it 155}
  \subitem {представление в виде упорядоченных списков} 157 (упр.~2.61)
\item {\texttt{adjoin-term}} 201, {\it 204}
\item {\texttt{advance-pc}} {\it 481}
\item {\texttt{after-delay}} 264, {\it 267}
\item {Algol (Алгол)}
  \subitem {бедность средств работы с составными объектами} 281{\it п}
  \subitem {блочная структура} 47
  \subitem {передача аргументов по имени [call by name]} 306{\it п}, 372{\it п}
  \subitem {санки} 306{\it п}, 372{\it п}
\item {\texttt{all-regs} (компилятор)} {\it 535}{\it п}
\item {\texttt{always-true}} {\it 434}
\item {\texttt{amb}} {\it 383}
\item {\texttt{ambeval}} {\it 396}
\item {\texttt{an-element-of}} {\it 383}
\item {\texttt{an-integer-starting-from}} {\it 383}
\item {\texttt{analyze}}
  \subitem {метациклическая} {\it 366}
  \subitem {недетерминистская} 396
\item {\texttt{analyze-amb}} {\it 401}
\item {\texttt{analyze-\ldots}}
  \subitem {метациклические} 366, 369 (упр.~4.23)
  \subitem {недетерминистские} 397
\item {\texttt{and} (особая форма)} 37
  \subitem {без подвыражений} 348 (упр.~4.4)
  \subitem {вычисление} 37
  \subitem {почему особая форма} 38
\item {\texttt{and} (язык запросов)} 411
  \subitem {обработка} 419, 432, 448 (упр.~4.76)
\item {\texttt{and-gate}} {\it 264}
\item {\texttt{angle}}
  \subitem {декартово представление} {\it 174}
  \subitem {полярное представление} {\it 175}
  \subitem {с помеченными данными} {\it 178}
  \subitem {управляемая данными} {\it 182}
\item {\texttt{angle-polar}} {\it 177}
\item {\texttt{angle-rectangular}} {\it 176}
\item {\texttt{announce-output}} {\it 356}
\item {APL} 125{\it п}
\item {\texttt{append}} 110, {\it 110}, {\it 244} (упр.~3.12)
  \subitem {vs. \texttt{append!}} 244 (упр.~3.12)
  \subitem {как накопление} 125 (упр.~2.33)
  \subitem {как регистровая машина} 494 (упр.~5.22)
  \subitem {с произвольным числом аргументов} 538{\it п}
  \subitem {<<что такое>> (правила) или <<как сделать>> (процедура)} 405
\item {\texttt{append!}} {\it 245} (упр.~3.12)
  \subitem {как регистровая машина} 494 (упр.~5.22)
\item {\texttt{append-instruction-sequences}} 521, {\it 537}
\item {\texttt{append-to-form} (правила)} {\it 415}
\item {\texttt{application?}} {\it 346}
\item {\texttt{apply} (ленивая)} {\it 373}
\item {\texttt{apply} (метациклическая)} {\it 341}
  \subitem {vs. элементарная \texttt{apply}} 355{\it п}
\item {\texttt{apply} (элементарная процедура)} {\it 182}{\it п}
\item {\texttt{apply-dispatch}} {\it 506}
  \subitem {с учетом скомпилированных процедур} {\it 551}
\item {\texttt{apply-generic}} {\it 182}
  \subitem {с башней типов} 194
  \subitem {с приведением} {\it 193}, 197 (упр.~2.81)
  \subitem {с приведением нескольких аргументов} 197 (упр.~2.82)
  \subitem {с приведением через последовательный подъем} 198 (упр.~2.84)
  \subitem {с упрощением типа} 198 (упр.~2.85)
  \subitem {через передачу сообщений} {\it 185}
\item {\texttt{apply-primitive-procedure}} 341, 351, {\it 356}
\item {\texttt{apply-rules}} {\it 436}
\item {\texttt{argl}, регистр} 501
\item {\texttt{articles}} {\it 389}
\item {\texttt{assemble}} {\it 476}, {\it 478}{\it п}
\item {\texttt{assert!} (интерпретатор запросов)} 424
\item {\texttt{assign} (в регистровой машине)} 455
  \subitem {имитация} 481
  \subitem {сохранение метки~в регистре} 461
\item {\texttt{assign-reg-name}} {\it 481}
\item {\texttt{assign-value-exp}} {\it 481}
\item {\texttt{assignment-value}} {\it 344}
\item {\texttt{assignment-variable}} {\it 344}
\item {\texttt{assignment?}} {\it 344}
\item {\texttt{assoc}} {\it 256}
\item {\texttt{atan} (элементарная процедура)} {\it 174}{\it п}
\item {\texttt{attach-tag}} {\it 176}
  \subitem {использование типов Scheme} 191 (упр.~2.78)
\item {\texttt{augend}} 150, {\it 152}
\item {\texttt{average}} {\it 41}
\item {\texttt{average-damp}} {\it 83}
\item {\texttt{averager} (ограничение)} 279 (упр.~3.33)
\bigskip
\item {B-деревья [B-trees]} 160{\it п}
\item {Basic (Бейсик)}
  \subitem {бедность средств работы с составными объектами} 281{\it п}
  \subitem {ограничения на составные данные} 107{\it п}
\item {\texttt{begin-actions}} {\it 346}
\item {\texttt{begin?}} {\it 346}
\item {\texttt{below}} 134, 145 (упр.~2.51)
\item {\texttt{beside}} 133, {\it 144}
\item {\texttt{branch} (в регистровой машине)} 454
  \subitem {имитация} 482
\item {\texttt{branch-dest}} {\it 482}
\bigskip
\item {C (Си)}
  \subitem {интерпретатор Scheme, написанный на Си} 557 (упр.~5.51), 557 (упр.~5.52)
  \subitem {компиляция процедур Scheme~в команды Си} 557 (упр.~5.52)
  \subitem {обработка ошибок} 516{\it п}, 554{\it п}
  \subitem {ограничения на составные данные} 107{\it п}
  \subitem {рекурсивные процедуры} 51
\item {\texttt{cadr}} 108{\it п}
\item {\texttt{ca\ldots{}r}} 108{\it п}
\item {\texttt{call-each}} {\it 266}
\item {\texttt{car} (элементарная процедура)} 95
  \subitem {как операция над списком} 108
  \subitem {описывающая аксиома} 100
  \subitem {происхождение имени} 95{\it п}
  \subitem {процедурная реализация} {\it 101}, {\it 102} (упр.~2.4), {\it 249}, {\it 379}
  \subitem {реализация с мутаторами} {\it 249}
  \subitem {реализация через векторы} 492
  \subitem {свойство замыкания} {\it 106}
\item {\texttt{cd\ldots{}r}} 108{\it п}
\item {\texttt{cdr} (элементарная процедура)} 95
  \subitem {как операция над списком} 108
  \subitem {описывающая аксиома} 100
  \subitem {происхождение имени} 95{\it п}
  \subitem {процедурная реализация} {\it 101}, 102 (упр.~2.4), {\it 249}, {\it 379}
  \subitem {реализация с мутаторами} {\it 249}
  \subitem {реализация через векторы} 492
\item {\texttt{celsius-fahrenheit-converter}} {\it 273}
  \subitem {в формате выражения} {\it 280} (упр.~3.37)
\item {\texttt{center}} {\it 104}
\item {\texttt{cesaro-stream}} {\it 330}
\item {\texttt{cesaro-test}} {\it 219}
\item {\texttt{coeff}} 201, {\it 204}
\item {Common Lisp} 24{\it п}
  \subitem {трактовка \texttt{nil}} 109{\it п}
\item {\texttt{compile}} {\it 520}
\item {\texttt{compile-and-go}} 551, {\it 553}
\item {\texttt{compile-and-run}} 556 (упр.~5.48)
\item {\texttt{compile-application}} {\it 530}
\item {\texttt{compile-assignment}} {\it 526}
\item {\texttt{compile-definition}} {\it 526}
\item {\texttt{compile-if}} {\it 527}
\item {\texttt{compile-lambda}} {\it 529}
\item {\texttt{compile-linkage}} {\it 524}
\item {\texttt{compile-proc-appl}} {\it 535}
\item {\texttt{compile-procedure-call}} {\it 533}
\item {\texttt{compile-quoted}} {\it 525}
\item {\texttt{compile-self-evaluating}} {\it 525}
\item {\texttt{compile-sequence}} {\it 528}
\item {\texttt{compile-variable}} {\it 525}
\item {\texttt{compiled-apply}} {\it 552}
\item {\texttt{compiled-procedure}} {\it 529}{\it п}
\item {\texttt{compiled-procedure-entry}} {\it 529}{\it п}
\item {\texttt{compiled-procedure-env}} {\it 529}{\it п}
\item {\texttt{complex}, пакет} {\it 189}
\item {\texttt{complex->complex}} {\it 197} (упр.~2.81)
\item {\texttt{compound-apply}} {\it 506}
\item {\texttt{compound-procedure?}} {\it 351}
\item {\texttt{cond} (особая форма)} {\it 36}
  \subitem {vs. \texttt{if}} 37{\it п}
  \subitem {вариант синтаксиса ветвей} {\it 349} (упр.~4.5)
  \subitem {ветвь} 36
  \subitem {вычисление} 36
  \subitem {неявный \texttt{begin} в следствиях} 214{\it п}
\item {\texttt{cond->if}} {\it 347}
\item {\texttt{cond-actions}} {\it 347}
\item {\texttt{cond-clauses}} {\it 347}
\item {\texttt{cond-else-clause?}} {\it 347}
\item {\texttt{cond-predicate}} {\it 347}
\item {\texttt{cond?}} {\it 347}
\item {\texttt{conjoin}} {\it 432}
\item {\texttt{connect}} 275, {\it 279}
\item {Conniver (Коннивер)} 385{\it п}
\item {\texttt{cons} (элементарная процедура)} 95
  \subitem {как операция над списком} 109
  \subitem {описывающие аксиомы} 100
  \subitem {происхождение имени} 95{\it п}
  \subitem {процедурная реализация} {\it 101}, {\it 101} (упр.~2.4), {\it 244}, {\it 249}, {\it 379}
  \subitem {реализация с мутаторами} {\it 249}
  \subitem {реализация через векторы} 493
\item {\texttt{cons-stream} (особая форма)} 301, {\it 302}
  \subitem {и ленивые вычисления} 379
  \subitem {почему особая форма} 303{\it п}
\item {\texttt{const} (в регистровой машине)} 455
  \subitem {имитация} 484
  \subitem {синтаксис} 471
\item {\texttt{constant} (элементарное ограничение)} {\it 277}
\item {\texttt{constant-exp-value}} {\it 484}
\item {\texttt{constant-exp?}} {\it 484}
\item {\texttt{construct-arglist}} {\it 531}
\item {\texttt{contents}} {\it 176}
  \subitem {использование типов Scheme} 191 (упр.~2.78)
\item {\texttt{continue}, регистр} {\it 461}
  \subitem {в вычислителе с явным управлением} {\it 501}
  \subitem {и рекурсия} {\it 467}
\item {\texttt{corner-split}} {\it 137}
\item {\texttt{cos} (элементарная процедура)} {\it 81}
\item {\texttt{count-change}} {\it 56}
\item {\texttt{count-leaves}} 115, {\it 117}
  \subitem {как накопление} 126 (упр.~2.35)
  \subitem {как регистровая машина} 494 (упр.~5.21)
\item {\texttt{count-pairs}} {\it 248} (упр.~3.16)
\item {\texttt{cube}} {\it 59} (упр.~1.15), {\it 69}, {\it 85}
\item {\texttt{cube-root}} {\it 84}
\item {\texttt{current-time}} 267, {\it 270}
\bigskip
\item {\texttt{decode}} {\it 167}
\item {\texttt{deep-reverse}} 117 (упр.~2.27)
\item {\texttt{define} (особая форма)} {\it 28}
  \subitem {vs. \texttt{lambda}} 75
  \subitem {внутренняя} {\it см.} внутренние определения
  \subitem {для процедур} {\it 32}, 75
  \subitem {значение выражения} 28{\it п}
  \subitem {модель с окружениями} 231
  \subitem {почему особая форма} 30
  \subitem {синтаксический сахар} 344
  \subitem {точечная запись} 112 (упр.~2.20)
\item {\texttt{define-variable!}} 351, {\it 353}
\item {\texttt{definition-value}} {\it 345}
\item {\texttt{definition-variable}} {\it 344}
\item {\texttt{definition?}} {\it 344}
\item {\texttt{delay} (особая форма)} {\it 302}
  \subitem {и ленивые вычисления} 379
  \subitem {мемоизированная} 305, 312 (упр.~3.57)
  \subitem {почему особая форма} 303{\it п}
  \subitem {реализация с помощью \texttt{lambda}} 305
  \subitem {явная} 325
  \subitem {явная vs. автоматическая} 380
\item {\texttt{delay-it}} {\it 375}
\item {\texttt{delete-queue!}} 251, {\it 253}
\item {\texttt{denom}} 94, {\it 96}
  \subitem {описывающая аксиома} {\it 100}
  \subitem {с приведением к наименьшему знаменателю} {\it 99}
\item {\texttt{deposit} (сообщение для банковского счета)} {\it 216}
\item {\texttt{deposit}, с внешним сериализатором} {\it 292}
\item {\texttt{deriv} (символическая)} {\it 151}
  \subitem {управляемая данными} {\it 183} (упр.~2.73)
\item {\texttt{deriv} (численная)} {\it 85}
\item {\texttt{disjoin}} {\it 432}
\item {\texttt{display} (элементарная процедура)} 67 (упр.~1.22), {\it 96}{\it п}
\item {\texttt{display-line}} {\it 302}
\item {\texttt{display-stream}} {\it 302}
\item {\texttt{distinct?}} {\it 387}{\it п}
\item {\texttt{div} (обобщенная)} {\it 187}
\item {\texttt{div-complex}} {\it 174}
\item {\texttt{div-interval}} {\it 103}
  \subitem {деление на ноль} 104 (упр.~2.10)
\item {\texttt{div-poly}} 205 (упр.~2.91)
\item {\texttt{div-rat}} {\it 94}
\item {\texttt{div-series}} 314 (упр.~3.62)
\item {\texttt{div-terms}} {\it 205} (упр.~2.91)
\item {\texttt{divides?}} {\it 64}
\item {\texttt{divisible?}} {\it 307}
\item {DOS/Windows} 516{\it п}
\item {\texttt{dot-product}} {\it 127} (упр.~2.37)
\item {\texttt{draw-line}} 141
\item {\texttt{driver-loop}}
  \subitem {для ленивого интерпретатора} {\it 374}
  \subitem {для метациклического интерпретатора} {\it 356}
  \subitem {для недетерминистского интерпретатора} {\it 402}
\bigskip
\item {$e$}
  \subitem {как решение дифференциального уравнения} 326
  \subitem {как цепная дробь} 83 (упр.~1.38)
\item {$e^x$, степенной ряд} 312 (упр.~3.59)
\item {\texttt{edge1-frame}} 140
\item {\texttt{edge2-frame}} 140
\item {EIEIO} 298{\it п}
\item {\texttt{element-of-set?}} 154
  \subitem {представление в виде бинарных деревьев} {\it 159}
  \subitem {представление в виде неупорядоченных списков} {\it 155}
  \subitem {представление в виде упорядоченных списков} {\it 156}
\item {\texttt{else} (особый символ~в \texttt{cond})} {\it 37}
\item {\texttt{empty-agenda?}} 267, {\it 270}
\item {\texttt{empty-arglist}} {\it 504}{\it п}
\item {\texttt{empty-instruction-sequence}} {\it 523}
\item {\texttt{empty-queue?}} 251, {\it 252}
\item {\texttt{empty-termlist?}} 201, {\it 204}
\item {\texttt{enclosing-environment}} {\it 352}
\item {\texttt{encode}} {\it 169} (упр.~2.68)
\item {\texttt{end-segment}} 99 (упр.~2.2), 142 (упр.~2.48)
\item {\texttt{end-with-linkage}} {\it 525}
\item {\texttt{entry}} {\it 159}
\item {\texttt{enumerate-interval}} {\it 123}
\item {\texttt{enumerate-tree}} {\it 123}
\item {\texttt{env}, регистр} 501
\item {\texttt{eq?} (элементарная процедура)} {\it 148}
  \subitem {для произвольных объектов} 247
  \subitem {и равенство чисел} 491{\it п}
  \subitem {как равенство указателей} 247, 492
  \subitem {реализация для символов} 492
\item {\texttt{equ?} (обобщенный предикат)} 191 (упр.~2.79)
\item {\texttt{equal-rat?}} {\it 95}
\item {\texttt{equal?}} 149 (упр.~2.54)
\item {\texttt{error} (элементарная процедура)} {\it 80}{\it п}
\item {\texttt{estimate-integral}} 221 (упр.~3.5)
\item {\texttt{estimate-pi}} {\it 219}
\item {\texttt{euler-transform}} {\it 316}
\item {\texttt{ev-application}} {\it 504}
\item {\texttt{ev-assignment}} {\it 510}
\item {\texttt{ev-begin}} {\it 507}
\item {\texttt{ev-definition}} {\it 511}
\item {\texttt{ev-if}} {\it 510}
\item {\texttt{ev-lambda}} {\it 503}
\item {\texttt{ev-quoted}} {\it 503}
\item {\texttt{ev-self-eval}} {\it 502}
\item {\texttt{ev-sequence}}
  \subitem {без хвостовой рекурсии} {\it 509}
  \subitem {с хвостовой рекурсией} {\it 508}
\item {\texttt{ev-variable}} {\it 502}
\item {\texttt{eval} (ленивая)} 373
\item {\texttt{eval} (метациклическая)} 339, {\it 340}
  \subitem {vs. элементарная \texttt{eval}} 359{\it п}
  \subitem {анализирующий вариант} {\it 366}
  \subitem {управляемая данными} 348 (упр.~4.3)
\item {\texttt{eval} (элементарная процедура)} {\it 359}
  \subitem {MIT Scheme} {\it 360}{\it п}
  \subitem {в интерпретаторе запросов} {\it 433}
\item {\texttt{eval-assignment}} {\it 342}
\item {\texttt{eval-definition}} {\it 343}
\item {\texttt{eval-dispatch}} {\it 502}
\item {\texttt{eval-if} (ленивая)} {\it 374}
\item {\texttt{eval-if} (метациклическая)} {\it 342}
\item {\texttt{eval-sequence}} {\it 342}
\item {\texttt{even-fibs}} {\it 120}, {\it 123}
\item {\texttt{even?}} {\it 60}
\item {\texttt{exchange}} {\it 292}
\item {\texttt{execute}} 475
\item {\texttt{execute-application}}
  \subitem {метациклическая} {\it 368}
  \subitem {недетерминистская} {\it 400}
\item {\texttt{exp}, регистр} 501
\item {\texttt{expand-clauses}} {\it 347}
\item {\texttt{expmod}} {\it 65}, {\it 68} (упр.~1.25), {\it 68} (упр.~1.26)
\item {\texttt{expt}}
  \subitem {линейно итеративный вариант} {\it 60}
  \subitem {линейно рекурсивный вариант} {\it 59}
  \subitem {регистровая машина} 470 (упр.~5.4)
\item {\texttt{extend-environment}} 351, {\it 352}
\item {\texttt{extend-if-consistent}} {\it 435}
\item {\texttt{extend-if-possible}} {\it 439}
\item {\texttt{external-entry}} {\it 553}
\item {\texttt{extract-labels}} {\it 478}, {\it 478}{\it п}
\bigskip
\item {\texttt{\#f}} 36{\it п}
\item {\texttt{factorial}}
  \subitem {использование стека, интерпретируемый вариант} 515 (упр.~5.26), 515 (упр.~5.27)
  \subitem {использование стека, регистровая машина} 488 (упр.~5.14)
  \subitem {использование стека, скомпилированный вариант} 555 (упр.~5.45)
  \subitem {как абстрактная машина} 357
  \subitem {компиляция} 539, 544
  \subitem {линейно итеративный вариант} {\it 50}
  \subitem {линейно рекурсивный вариант} {\it 49}
  \subitem {регистровая машина (итеративная)} 454 (упр.~5.1), 456 (упр.~5.2)
  \subitem {регистровая машина (рекурсивная)} 466, 469
  \subitem {с присваиванием} {\it 226}
  \subitem {структура окружений при вычислении} 233 (упр.~3.9)
  \subitem {через процедуры высших порядков} 73 (упр.~1.31)
\item {\texttt{false}} {\it 36}{\it п}
\item {\texttt{false?}} {\it 350}
\item {\texttt{fast-expt}} {\it 60}
\item {\texttt{fast-prime}} {\it 66}
\item {\texttt{fermat-test}} {\it 66}
\item {\texttt{fetch-assertions}} {\it 440}
\item {\texttt{fetch-rules}} {\it 440}
\item {\texttt{fib}}
  \subitem {древовидно-рекурсивный вариант} {\it 53}, {\it 516} (упр.~5.29)
  \subitem {использование стека, интерпретируемый вариант} 516 (упр.~5.29)
  \subitem {использование стека, скомпилированный вариант} 555 (упр.~5.46)
  \subitem {линейно-итеративный вариант} 55
  \subitem {логарифмический вариант} 62 (упр.~1.19)
  \subitem {регистровая машина (с древовидной рекурсией)} 468, 470
  \subitem {с именованным \texttt {let}} {\it 350} (упр.~4.8)
  \subitem {с мемоизацией} {\it 260} (упр.~3.27)
\item {\texttt{fibs} (бесконечный поток)} {\it 308}
  \subitem {неявное определение} {\it 310}
\item {FIFO} 250
\item {\texttt{filter}} {\it 122}
\item {\texttt{filtered-accumulate}} 74 (упр.~1.33)
\item {\texttt{find-assertions}} {\it 434}
\item {\texttt{find-divisor}} {\it 64}
\item {\texttt{first-agenda-item}} 267, {\it 271}
\item {\texttt{first-exp}} {\it 346}
\item {\texttt{first-frame}} {\it 352}
\item {\texttt{first-operand}} {\it 346}
\item {\texttt{first-segment}} {\it 270}
\item {\texttt{first-term}} 201, {\it 204}
\item {\texttt{fixed-point}} {\it 80}
  \subitem {как пошаговое улучшение} 89 (упр.~1.46)
\item {\texttt{fixed-point-of-transform}} {\it 86}
\item {\texttt{flag}, регистр} 475
\item {\texttt{flatmap}} {\it 129}
\item {\texttt{flatten-stream}} {\it 443}
\item {\texttt{flip-horiz}} 134, 144 (упр.~2.50)
\item {\texttt{flip-vert}} 134, {\it 143}
\item {\texttt{flipped-pairs}} {\it 136}, {\it 139}, {\it 139}{\it п}
\item {\texttt{fold-left}} {\it 127} (упр.~2.38)
\item {\texttt{fold-right}} 127 (упр.~2.38)
\item {\texttt{for-each}} 114 (упр.~2.23), {\it 377} (упр.~4.30)
\item {\texttt{for-each-except}} {\it 279}
\item {\texttt{force}} 302, {\it 305}
  \subitem {vs. вынуждение санка} 372{\it п}
\item {\texttt{force-it}} {\it 375}
  \subitem {вариант с мемоизацией} {\it 376}
\item {\texttt{forget-value!}} 275, {\it 279}
\item {Fortran (Фортран)} 24, 125{\it п}
  \subitem {изобретатель} 333{\it п}
  \subitem {ограничения на составные данные} 107{\it п}
\item {\texttt{frame-coord-map}} {\it 140}
\item {\texttt{frame-values}} {\it 352}
\item {\texttt{frame-variables}} {\it 352}
\item {Franz Lisp (Франц Лисп)} 24{\it п}
\item {\texttt{free}, регистр} 493, 496
\item {\texttt{fringe}} 118 (упр.~2.28)
  \subitem {как перечисление листьев дерева} 123{\it п}
\item {\texttt{front-ptr}} {\it 251}
\item {\texttt{front-queue}} 251, {\it 252}
\item {\texttt{full-adder}} {\it 263}
\bigskip
\item {\texttt{gcd}} {\it 63}
  \subitem {регистровая машина} 451, 472
\item {\texttt{gcd-terms}} {\it 208}
\item {\texttt{generate-huffman-tree}} {\it 169} (упр.~2.69)
\item {\texttt{get}} 180, {\it 259}
\item {\texttt{get-contents}} {\it 474}
\item {\texttt{get-global-environment}} {\it 512}{\it п}
\item {\texttt{get-register}} {\it 476}
\item {\texttt{get-register-contents}} 472, {\it 475}
\item {\texttt{get-signal}} 264, {\it 266}
\item {\texttt{get-value}} 275, {\it 279}
\item {\texttt{goto} (в регистровой машине)} 454
  \subitem {имитация} 482
  \subitem {переход на содержимое регистра} 461
\item {\texttt{goto-dest}} {\it 483}
\bigskip
\item {\texttt{half-adder}} {\it 262}
\item {\texttt{half-interval-method}} {\it 80}
\item {\texttt{has-value?}} 275, {\it 279}
\item {Hassle (Закавыка)} 371{\it п}
\bigskip
\item {IBM 704} 95{\it п}
\item {\texttt{identity}} {\it 72}
\item {\texttt{if} (особая форма)} {\it 37}
  \subitem {vs. \texttt{cond}} 37{\it п}
  \subitem {вычисление} 37
  \subitem {нормальный порядок вычисления формы} 39 (упр.~1.5)
  \subitem {одностороннее предложение (без альтернативы)} 271{\it п}
  \subitem {почему особая форма} 42 (упр.~1.6)
  \subitem {предикат, следствие~и альтернатива} 37
\item {\texttt{if-alternative}} {\it 345}
\item {\texttt{if-consequent}} {\it 345}
\item {\texttt{if-predicate}} {\it 345}
\item {\texttt{if?}} {\it 345}
\item {\texttt{imag-part}}
  \subitem {декартово представление} {\it 174}
  \subitem {полярное представление} {\it 175}
  \subitem {с помеченными данными} {\it 177}
  \subitem {управляемая данными} {\it 182}
\item {\texttt{imag-part-polar}} {\it 177}
\item {\texttt{imag-part-rectangular}} {\it 176}
\item {\texttt{inc}} {\it 71}
\item {\texttt{inform-about-no-value}} {\it 276}
\item {\texttt{inform-about-value}} {\it 276}
\item {\texttt{initialize-stack}, операция регистровой машины} 474, 493
\item {\texttt{insert!}}
  \subitem {для двумерной таблицы} {\it 258}
  \subitem {для одномерной таблицы} {\it 256}
\item {\texttt{insert-queue!}} 251, {\it 252}
\item {\texttt{install-complex-package}} {\it 189}
\item {\texttt{install-polar-package}} {\it 181}
\item {\texttt{install-polynomial-package}} {\it 200}
\item {\texttt{install-rational-package}} {\it 188}
\item {\texttt{install-rectangular-package}} {\it 181}
\item {\texttt{install-scheme-number-package}} {\it 187}
\item {\texttt{instantiate}} {\it 431}
\item {\texttt{instruction-execution-proc}} {\it 479}
\item {\texttt{instruction-text}} {\it 479}
\item {\texttt{integers} (бесконечный поток)} {\it 307}
  \subitem {вариант с ленивыми списками} {\it 380}
  \subitem {неявное определение} {\it 309}
\item {\texttt{integers-starting-from}} {\it 307}
\item {\texttt{integral}} {\it 72}, {\it 322}, {\it 326} (упр.~3.77)
  \subitem {вариант с ленивыми списками} {\it 380}
  \subitem {необходимость задержанного вычисления} 325
  \subitem {с задержанным аргументом} {\it 325}
  \subitem {с использованием \texttt {lambda}} {\it 75}
\item {\texttt{integrate-series}} 313 (упр.~3.59)
\item {\texttt{interleave}} {\it 319}
\item {\texttt{interleave-delayed}} {\it 442}
\item {InterLisp (ИнтерЛисп)} 24{\it п}
\item {\texttt{intersection-set}} 154
  \subitem {представление в виде бинарных деревьев} 162 (упр.~2.65)
  \subitem {представление в виде неупорядоченных списков} {\it 155}
  \subitem {представление в виде упорядоченных списков} {\it 157}
\item {\texttt{inverter}} {\it 264}
\bigskip
\item {KRC} 128{\it п}, 319{\it п}
\bigskip
\item {\texttt{label} (в регистровой машине)} 454
  \subitem {имитация} 484
\item {\texttt{label-exp-label}} {\it 484}
\item {\texttt{label-exp?}} {\it 484}
\item {\texttt{lambda} (особая форма)} {\it 74}
  \subitem {vs. \texttt{define}} 75
  \subitem {точечная запись} 112{\it п}
\item {\texttt{lambda-body}} {\it 345}
\item {\texttt{lambda-parameters}} {\it 345}
\item {\texttt{lambda-выражение}}
  \subitem {значение} 231
\item {\texttt{lambda?}} {\it 345}
\item {\texttt{last-operand?}} {\it 504}{\it п}
\item {\texttt{last-pair}} 111 (упр.~2.17), {\it 245} (упр.~3.12)
  \subitem {правила} 416 (упр.~4.62)
\item {\texttt{leaf?}} {\it 166}
\item {\texttt{left-branch}} {\it 159}, {\it 167}
\item {\texttt{length}} 110
  \subitem {итеративный вариант} {\it 110}
  \subitem {как накопление} 125 (упр.~2.33)
  \subitem {рекурсивный вариант} {\it 110}
\item {\texttt{let} (особая форма)} {\it 76}
  \subitem {vs. внутреннее определение} 78
  \subitem {именованный} {\it 349} (упр.~4.8)
  \subitem {как синтаксический сахар} 77, 238 (упр.~3.10)
  \subitem {модель вычисления} 238 (упр.~3.10)
  \subitem {область действия переменных} 77
\item {\texttt{let*} (особая форма)} {\it 349} (упр.~4.7)
\item {\texttt{letrec} (особая форма)} {\it 363} (упр.~4.20)
\item {\texttt{lexical-address-lookup}} 548, 549 (упр.~5.39)
\item {Lisp (Лисп)}
  \subitem {vs. Паскаль} 31{\it п}
  \subitem {vs. Фортран} 24
  \subitem {аппликативный порядок вычислений} 35
  \subitem {диалекты} {\it см.} диалекты Лиспа
  \subitem {история} 23
  \subitem {исходная реализация на IBM 704} 95{\it п}
  \subitem {на DEC PDP-1} 495{\it п}
  \subitem {процедуры как объекты первого класса} 87
  \subitem {система внутренних типов} 191 (упр.~2.78)
  \subitem {сокращение от LISt Processing} 23
  \subitem {удобство для написания вычислителей} 337
  \subitem {уникальные свойства} 24
  \subitem {эффективность} 24, 27{\it п}
\item {\texttt{lisp-value} (интерпретатор запросов)} {\it 433}
\item {\texttt{lisp-value} (язык запросов)} 412, 427
  \subitem {вычисление} 421, 433, 448 (упр.~4.77)
\item {\texttt{list} (элементарная процедура)} {\it 108}
\item {\texttt{list->tree}} {\it 161} (упр.~2.64)
\item {\texttt{list-difference}} {\it 537}
\item {\texttt{list-exp?}} {\it 346}
\item {\texttt{list-of-arg-values}} {\it 374}
\item {\texttt{list-of-delayed-args}} {\it 374}
\item {\texttt{list-of-values}} {\it 341}
\item {\texttt{list-ref}} {\it 109}, {\it 379}
\item {\texttt{list-union}} {\it 537}
\item {\texttt{log} (элементарная процедура)} {\it 82} (упр.~1.36)
\item {\texttt{logical-not}} {\it 264}
\item {\texttt{lookup}}
  \subitem {в двумерной таблице} {\it 258}
  \subitem {в множестве записей} {\it 162}
  \subitem {в одномерной таблице} {\it 255}
\item {\texttt{lookup-label}} {\it 479}
\item {\texttt{lookup-prim}} {\it 485}
\item {\texttt{lookup-variable-value}} 351, {\it 352}
  \subitem {при исключенных внутренних определениях} 362 (упр.~4.16)
\item {\texttt{lower-bound}} 103 (упр.~2.7)
\bigskip
\item {\texttt{machine}} {\it 484}
\item {Macintosh} 516{\it п}
\item {MacLisp (МакЛисп)} 24{\it п}
\item {\texttt{magnitude}}
  \subitem {декартово представление} {\it 174}
  \subitem {полярное представление} {\it 175}
  \subitem {с помеченными данными} {\it 177}
  \subitem {управляемая данными} {\it 182}
\item {\texttt{magnitude-polar}} {\it 177}
\item {\texttt{magnitude-rectangular}} {\it 176}
\item {\texttt{make-account}} {\it 216}
  \subitem {в модели с окружениями} 240 (упр.~3.11)
  \subitem {с сериализацией} {\it 289}, {\it 290} (упр.~3.41), {\it 291} (упр.~3.42)
\item {\texttt{make-account-and-serializer}} {\it 292}
\item {\texttt{make-accumulator}} 217 (упр.~3.1)
\item {\texttt{make-agenda}} 267, {\it 270}
\item {\texttt{make-assign}} {\it 481}
\item {\texttt{make-begin}} {\it 346}
\item {\texttt{make-branch}} {\it 118} (упр.~2.29), {\it 482}
\item {\texttt{make-center-percent}} {\it 104} (упр.~2.12)
\item {\texttt{make-center-width}} {\it 104}
\item {\texttt{make-code-tree}} {\it 166}
\item {\texttt{make-compiled-procedure}} {\it 529}{\it п}
\item {\texttt{make-complex-from-mag-ang}} {\it 190}
\item {\texttt{make-complex-from-real-imag}} {\it 189}
\item {\texttt{make-connector}} {\it 278}
\item {\texttt{make-cycle}} {\it 245} (упр.~3.13)
\item {\texttt{make-decrementer}} {\it 222}
\item {\texttt{make-execution-procedure}} {\it 480}
\item {\texttt{make-frame}} 140, {\it 141} (упр.~2.47), {\it 352}
\item {\texttt{make-from-mag-ang}} {\it 178}, {\it 183}
  \subitem {в виде передачи сообщений} 185 (упр.~2.75)
  \subitem {декартово представление} {\it 174}
  \subitem {полярное представление} {\it 175}
\item {\texttt{make-from-mag-ang-polar}} {\it 177}
\item {\texttt{make-from-mag-ang-rectangular}} {\it 177}
\item {\texttt{make-from-real-imag}} {\it 178}, {\it 183}
  \subitem {в виде передачи сообщений} 185
  \subitem {декартово представление} {\it 174}
  \subitem {полярное представление} {\it 175}
\item {\texttt{make-from-real-imag-polar}} {\it 177}
\item {\texttt{make-from-real-imag-rectangular}} {\it 176}
\item {\texttt{make-goto}} {\it 482}
\item {\texttt{make-if}} {\it 345}
\item {\texttt{make-instruction}} {\it 479}
\item {\texttt{make-instruction-sequence}} {\it 523}
\item {\texttt{make-interval}} 103, {\it 103} (упр.~2.7)
\item {\texttt{make-joint}} 226 (упр.~3.7)
\item {\texttt{make-label}} {\it 527}{\it п}
\item {\texttt{make-label-entry}} {\it 479}
\item {\texttt{make-lambda}} {\it 345}
\item {\texttt{make-leaf}} {\it 166}
\item {\texttt{make-leaf-set}} {\it 168}
\item {\texttt{make-machine}} 472, {\it 473}
\item {\texttt{make-monitored}} 217 (упр.~3.2)
\item {\texttt{make-mutex}} {\it 295}
\item {\texttt{make-new-machine}} {\it 477}
\item {\texttt{make-operation-exp}} {\it 484}
\item {\texttt{make-perform}} {\it 483}
\item {\texttt{make-point}} 99 (упр.~2.2)
\item {\texttt{make-poly}} {\it 200}
\item {\texttt{make-polynomial}} {\it 204}
\item {\texttt{make-procedure}} {\it 351}
\item {\texttt{make-product}} 150, {\it 151}
\item {\texttt{make-queue}} 251, {\it 252}
\item {\texttt{make-rat}} 94, {\it 96}, {\it 98}
  \subitem {описывающая аксиома} {\it 100}
  \subitem {приведение к наименьшему знаменателю} {\it 97}
\item {\texttt{make-rational}} {\it 188}
\item {\texttt{make-register}} {\it 474}
\item {\texttt{make-restore}} {\it 483}
\item {\texttt{make-save}} {\it 483}
\item {\texttt{make-scheme-number}} {\it 188}
\item {\texttt{make-segment}} 99 (упр.~2.2), 142 (упр.~2.48)
\item {\texttt{make-serializer}} {\it 295}
\item {\texttt{make-simplified-withdraw}} {\it 222}, {\it 332}
\item {\texttt{make-stack}} {\it 474}
  \subitem {с отслеживанием стека} {\it 487}
\item {\texttt{make-sum}} 150, {\it 151}
\item {\texttt{make-table}}
  \subitem {одномерная таблица} {\it 256}
  \subitem {реализация через передачу сообщений} {\it 258}
\item {\texttt{make-tableau}} {\it 316}
\item {\texttt{make-term}} 201, {\it 204}
\item {\texttt{make-test}} {\it 482}
\item {\texttt{make-time-segment}} {\it 269}
\item {\texttt{make-tree}} {\it 159}
\item {\texttt{make-vect}} 141 (упр.~2.46)
\item {\texttt{make-wire}} 262, {\it 265}, 269 (упр.~3.31)
\item {\texttt{make-withdraw}} {\it 215}
  \subitem {в модели с окружениями} 234
  \subitem {с использованием \texttt {let}} {\it 237} (упр.~3.10)
\item {\texttt{map}} {\it 113}, {\it 379}
  \subitem {как накопление} 125 (упр.~2.33)
  \subitem {с несколькими аргументами} 113{\it п}
\item {\texttt{map-over-symbols}} {\it 444}
\item {\texttt{map-successive-pairs}} {\it 330}
\item {\texttt{matrix-*-matrix}} 127 (упр.~2.37)
\item {\texttt{matrix-*-vector}} 127 (упр.~2.37)
\item {\texttt{max} (элементарная процедура)} {\it 103}
\item {MDL} 496{\it п}
\item {\texttt{member}} 387{\it п}
\item {\texttt{memo-fib}} {\it 260} (упр.~3.27)
\item {\texttt{memo-proc}} {\it 305}
\item {\texttt{memoize}} {\it 260} (упр.~3.27)
\item {\texttt{memq}} {\it 148}
\item {\texttt{merge}} {\it 311} (упр.~3.56)
\item {\texttt{merge-weighted}} 320 (упр.~3.70)
\item {MicroPlanner (МикроПлэнер)} 385{\it п}
\item {\texttt{min} (элементарная процедура)} {\it 103}
\item {Miranda (Миранда)} 128{\it п}
\item {MIT} 405{\it п}
  \subitem {Исследовательская лаборатория по Электронике} 23, 495{\it п}
  \subitem {лаборатория Искусственного Интеллекта} 24{\it п}
  \subitem {проект MAC} 24{\it п}
  \subitem {ранняя история} 133{\it п}
\item {MIT Scheme}
  \subitem {\texttt{eval}} 360{\it п}
  \subitem {\texttt{random}} 221{\it п}
  \subitem {\texttt{user-initial-environment}} 360{\it п}
  \subitem {{\tt without-interrupts}} 296{\it п}
  \subitem {внутренние определения} 362{\it п}
  \subitem {пустой поток} 301{\it п}
  \subitem {числа} 42{\it п}
\item {ML} 329{\it п}
\item {\texttt{modifies-register?}} {\it 536}
\item {modus ponens} 425{\it п}
\item {\texttt{monte-carlo}} {\it 219}
  \subitem {бесконечный поток} {\it 331}
\item {\texttt{mul} (обобщенная)} {\it 187}
  \subitem {примененная к коэффициентам многочленов} 202
\item {\texttt{mul-complex}} {\it 174}
\item {\texttt{mul-interval}} {\it 103}
  \subitem {более эффективная версия} {\it 104} (упр.~2.11)
\item {\texttt{mul-poly}} {\it 200}
\item {\texttt{mul-rat}} {\it 94}
\item {\texttt{mul-series}} 313 (упр.~3.60)
\item {\texttt{mul-streams}} 311 (упр.~3.54)
\item {\texttt{mul-terms}} {\it 202}
\item {Multics, система разделения времени [Multics time-sharing system]} 495{\it п}
\item {\texttt{multiple-dwelling}} {\it 387}
\item {\texttt{multiplicand}} {\it 152}
\item {\texttt{multiplier}}
  \subitem {селектор} {\it 152}
  \subitem {элементарное ограничение} {\it 276}
\item {\texttt{mystery}} {\it 245} (упр.~3.14)
\bigskip
\item {n-кратно сглаженная функция [n-fold smoothed function]} 88 (упр.~1.44)
\item {\texttt{needs-register?}} {\it 536}
\item {\texttt{negate}} {\it 433}
\item {\texttt{new}, регистр} 498
\item {\texttt{new-cars}, регистр} 496
\item {\texttt{new-cdrs}, регистр} 496
\item {\texttt{new-withdraw}} {\it 214}
\item {\texttt{newline} (элементарная процедура)} 67 (упр.~1.22), {\it 96}{\it п}
\item {\texttt{newton-transform}} {\it 85}
\item {\texttt{newtons-method}} {\it 85}
\item {\texttt{next} (описатель связи)} 521
\item {\texttt{next-to} (правила)} {\it 416} (упр.~4.61)
\item {\texttt{nil}}
  \subitem {избавление от} 148
  \subitem {как обыкновенная переменная в Scheme} 109{\it п}
  \subitem {как показатель конца списка} 107
  \subitem {как пустой список} 109
\item {\texttt{no-more-exps?}} {\it 509}{\it п}
\item {\texttt{no-operands?}} {\it 346}
\item {\texttt{not} (элементарная процедура)} {\it 37}
\item {\texttt{not} (язык запросов)} 412, 427
  \subitem {вычисление} 421, 433, 448 (упр.~4.77)
\item {\texttt{null?} (элементарная процедура)} {\it 110}
  \subitem {реализация через типизированные указатели} 493
\item {\texttt{number?} (элементарная процедура)} {\it 150}
  \subitem {и тип данных} 191 (упр.~2.78)
  \subitem {реализация через типизированные указатели} 493
\item {\texttt{numer}} 94, {\it 96}
  \subitem {описывающая аксиома} {\it 100}
  \subitem {с приведением к наименьшему знаменателю} {\it 98}
\bigskip
\item {\texttt{old}, регистр} 498
\item {\texttt{oldcr}, регистр} 499
\item {\texttt{ones} (бесконечный поток)}
  \subitem {вариант с ленивыми списками} {\it 380}
\item {\texttt{op} (в регистровой машине)} 455
  \subitem {имитация} 484
\item {\texttt{operands}} {\it 183} (упр.~2.73), {\it 346}
\item {\texttt{operation-exp-op}} {\it 485}
\item {\texttt{operation-exp-operands}} {\it 485}
\item {\texttt{operation-exp?}} {\it 485}
\item {\texttt{operation-table}} {\it 259}
\item {\texttt{operator}} {\it 183} (упр.~2.73), {\it 346}
\item {\texttt{or} (особая форма)} 37
  \subitem {без подвыражений} 348 (упр.~4.4)
  \subitem {вычисление} 37
  \subitem {почему особая форма} 38
\item {\texttt{or} (язык запросов)} 411
  \subitem {обработка} 420, 432
\item {\texttt{or-gate}} 265 (упр.~3.28), 265 (упр.~3.29)
\item {\texttt{order}} 201, {\it 204}
\item {\texttt{origin-frame}} 140
\bigskip
\item {P-операция на семафоре} 294{\it п}
\item {\texttt{pair?} (элементарная процедура)} {\it 117}
  \subitem {реализация через типизированные указатели} 493
\item {\texttt{pairs}} {\it 319}
\item {\texttt{parallel-execute}} 288
\item {\texttt{parallel-instruction-sequences}} {\it 539}
\item {\texttt{parse}} {\it 390}
\item {\texttt{parse-\ldots}} 389
\item {\texttt{partial-sums}} 311 (упр.~3.55)
\item {Pascal (Паскаль)} 31{\it п}
  \subitem {бедность средств работы с составными объектами} 281{\it п}
  \subitem {ограничения на составные данные} 107{\it п}
  \subitem {рекурсивные процедуры} 51
  \subitem {сложности с процедурами высших порядков} 328{\it п}
\item {\texttt{pattern-match}} {\it 435}
\item {\texttt{pc}, регистр} 475
\item {\texttt{perform} (в регистровой машине)} 457
  \subitem {имитация} 483
\item {\texttt{perform-action}} {\it 484}
\item {\texttt{permutations}} {\it 129}
\item {\texttt{pi-stream}} {\it 315}
\item {\texttt{pi-sum}} {\it 71}
  \subitem {через \texttt {lambda}} {\it 75}
  \subitem {через процедуры высших порядков} {\it 72}
\item {Planner (Плэнер)} 385{\it п}
\item {\texttt{polar}, пакет} 181
\item {\texttt{polar?}} {\it 176}
\item {poly} 200
\item {\texttt{polynomial}, пакет} 200
\item {\texttt{pop}} {\it 475}
\item {Portable Standard Lisp (Переносимый Стандартный Лисп)} 24{\it п}
\item {PowerPC} 298{\it п}
\item {\texttt{prepositions}} {\it 391}
\item {\texttt{preserving}} 521, 523 (упр.~5.31), {\it 538}, 546 (упр.~5.37)
\item {\texttt{prime}} {\it 310}
\item {\texttt{prime-sum-pair}} {\it 382}
\item {\texttt{prime-sum-pairs}} {\it 129}
  \subitem {бесконечный поток} 318
\item {\texttt{prime?}} {\it 64}
\item {\texttt{primes} (бесконечный поток)} {\it 308}
  \subitem {неявное определение} {\it 310}
\item {\texttt{primitive-apply}} {\it 506}
\item {\texttt{primitive-implementation}} {\it 355}
\item {\texttt{primitive-procedure-names}} {\it 355}
\item {\texttt{primitive-procedure-objects}} {\it 355}
\item {\texttt{primitive-procedure?}} 351, {\it 355}
\item {\texttt{print}, операция~в регистровой машине} 457
\item {\texttt{print-point}} {\it 99} (упр.~2.2)
\item {\texttt{print-queue}} 254 (упр.~3.21)
\item {\texttt{print-rat}} {\it 96}
\item {\texttt{print-result}} {\it 512}
  \subitem {с отслеживаемым количеством стековых операций} 514
\item {\texttt{probe}}
  \subitem {в имитаторе цифровых схем} {\it 268}
  \subitem {в системе ограничений} {\it 277}
\item {\texttt{proc}, регистр} 501
\item {\texttt{procedure-body}} {\it 351}
\item {\texttt{procedure-environment}} {\it 351}
\item {\texttt{procedure-parameters}} {\it 351}
\item {\texttt{product}} 73 (упр.~1.31)
  \subitem {как накопление} 74 (упр.~1.32)
\item {\texttt{product?}} 150, {\it 152}
\item {Prolog (Пролог)} 385{\it п}, 405{\it п}
\item {\texttt{prompt-for-input}} {\it 356}
\item {\texttt{propagate}} {\it 267}
\item {\texttt{push}} {\it 475}
\item {\texttt{put}} 180, {\it 259}
\bigskip
\item {\texttt{qeval}} 424, {\it 431}
\item {\texttt{queens}} {\it 130} (упр.~2.42)
\item {\texttt{query-driver-loop}} {\it 430}
\item {\texttt{quote} (особая форма)} 147{\it п}
  \subitem {и \texttt{read}} 356{\it п}, 444{\it п}
\item {\texttt{quoted?}} {\it 344}
\item {\texttt{quotient} (элементарная процедура)} 312 (упр.~3.58)
\bigskip
\item {\texttt{rand}} {\it 219}
  \subitem {со сбрасыванием} 221 (упр.~3.6)
\item {\texttt{random} (элементарная процедура)} 66
  \subitem {MIT Scheme} 221{\it п}
  \subitem {необходимость присваивания} 213{\it п}
\item {\texttt{random-in-range}} {\it 221} (упр.~3.5)
\item {\texttt{rational-package} (пакет)} {\it 188}
\item {RC-цепь [RC circuit]} 322 (упр.~3.73)
\item {\texttt{read} (элементарная процедура)} {\it 356}{\it п}
  \subitem {макросимволы ввода} 444{\it п}
  \subitem {обработка точечной записи} 436
\item {\texttt{read}, операция регистровой машины} 456
\item {\texttt{read-eval-print} loop} {\it 512}
\item {\texttt{real-part}}
  \subitem {декартово представление} {\it 174}
  \subitem {полярное представление} {\it 175}
  \subitem {с помеченными данными} {\it 177}
  \subitem {управляемая данными} {\it 182}
\item {\texttt{real-part-polar}} {\it 177}
\item {\texttt{real-part-rectangular}} {\it 176}
\item {\texttt{rear-ptr}} {\it 251}
\item {\texttt{receive}, процедура} 476
\item {\texttt{rectangular}, пакет} 181
\item {\texttt{rectangular?}} {\it 176}
\item {\texttt{reg} (в регистровой машине)} 455
  \subitem {имитация} 484
\item {\texttt{register-exp-reg}} {\it 484}
\item {\texttt{register-exp?}} {\it 484}
\item {\texttt{registers-modified}} {\it 536}
\item {\texttt{registers-needed}} {\it 536}
\item {\texttt{remainder} (элементарная процедура)} {\it 60}
\item {\texttt{remainder-terms}} 208 (упр.~2.94)
\item {\texttt{remove}} {\it 130}
\item {\texttt{remove-first-agenda-item!}} 267
\item {\texttt{require}} {\it 383}
  \subitem {как особая форма} 403 (упр.~4.54)
\item {\texttt{rest-exps}} {\it 346}
\item {\texttt{rest-operands}} {\it 346}
\item {\texttt{rest-segments}} {\it 270}
\item {\texttt{rest-terms}} 201, {\it 204}
\item {\texttt{restore} (в регистровой машине)} 467
  \subitem {моделирование} 483
  \subitem {реализация} 493
\item {\texttt{return} (описатель связи)} 521
\item {\texttt{reverse}} 111 (упр.~2.18)
  \subitem {как свертка} 128 (упр.~2.39)
  \subitem {правила} 429 (упр.~4.68)
\item {\texttt{right-branch}} {\it 159}, {\it 167}
\item {\texttt{right-split}} {\it 136}
\item {RLC-цепь последовательная [series RLC circuit]} 328 (упр.~3.80)
\item {\texttt{root}, регистр} 496
\item {\texttt{rotate90}} {\it 144}
\item {\texttt{round} (элементарная процедура)} {\it 198}{\it п}
\item {RSA алгоритм [RSA algorithm]} 67{\it п}
\item {\texttt{runtime} (элементарная процедура)} {\it 67} (упр.~1.22)
\bigskip
\item {\texttt{same} (правило)} {\it 413}
\item {\texttt{same-variable?}} 150, {\it 151}, 200
\item {\texttt{save} (в регистровой машине)} 467
  \subitem {моделирование} 483
  \subitem {реализация} 493
\item {\texttt{scale-list}} {\it 112}, {\it 113}, {\it 380}
\item {\texttt{scale-stream}} {\it 310}
\item {\texttt{scale-tree}} {\it 119}
\item {\texttt{scale-vect}} 141 (упр.~2.46)
\item {\texttt{scan}, регистр} 496
\item {\texttt{scan-out-defines}} 362 (упр.~4.16)
\item {Scheme (Схема)} 24
  \subitem {история} 24{\it п}
\item {\texttt{search}} {\it 79}
\item {\texttt{segment-queue}} {\it 269}
\item {\texttt{segment-time}} {\it 269}
\item {\texttt{segments}} {\it 270}
\item {\texttt{segments->painter}} {\it 142}
\item {\texttt{self-evaluating?}} {\it 343}
\item {\texttt{sequence->exp}} {\it 346}
\item {\texttt{serialized-exchange}} {\it 293}
  \subitem {с избежанием тупиков} 297 (упр.~3.48)
\item {\texttt{set!} (особая форма)} 214, {\it см. также} присваивание
  \subitem {значение} 214{\it п}
  \subitem {модель с окружениями} 231{\it п}
\item {\texttt{set-car!}} {\it 242}
\item {\texttt{set-cdr!}} {\it 242}
\item {\texttt{set-contents!}} {\it 474}
\item {\texttt{set-current-time!}} {\it 270}
\item {\texttt{set-front-ptr!}} {\it 251}
\item {\texttt{set-instruction-execution-proc!}} {\it 479}
\item {\texttt{set-rear-ptr!}} {\it 251}
\item {\texttt{set-register-contents!}} 472, {\it 476}
\item {\texttt{set-segments!}} {\it 270}
\item {\texttt{set-signal!}} 264, {\it 266}
\item {\texttt{set-value!}} 275, {\it 279}
\item {\texttt{set-variable-value!}} 351, {\it 353}
\item {\texttt{setup-environment}} {\it 354}
\item {\texttt{shrink-to-upper-right}} {\it 143}
\item {\texttt{signal-error}} {\it 513}
\item {\texttt{simple-query}} {\it 431}
\item {\texttt{sin} (элементарная процедура)} {\it 81}
\item {\texttt{singleton-stream}} {\it 443}
\item {SKETCHPAD} 272{\it п}
\item {\texttt{smallest-divisor}} {\it 64}
  \subitem {более эффективный вариант} 68 (упр.~1.23)
\item {Smalltalk} 272{\it п}
\item {\texttt{solve}} {\it 325}, {\it 326}
  \subitem {вариант с ленивыми списками} {\it 380}
  \subitem {с прочесанными внутренними определениями} {\it 363} (упр.~4.18)
\item {\texttt{split}} 139 (упр.~2.45)
\item {\texttt{sqare}} {\it 31}
\item {\texttt{sqare-limit}} {\it 137}, {\it 139}
\item {\texttt{sqrt}} {\it 42}
  \subitem {в модели с окружениями} {\it 238}
  \subitem {как неподвижная точка} {\it 81}, {\it 84}, {\it 86}
  \subitem {как пошаговое улучшение} 89 (упр.~1.46)
  \subitem {как предел потока} 317 (упр.~3.64)
  \subitem {регистровая машина} 459 (упр.~5.3)
  \subitem {с блочной структурой} {\it 47}
  \subitem {с методом Ньютона} {\it 86}
\item {\texttt{sqrt-stream}} {\it 314}
\item {\texttt{square}}
  \subitem {в модели с окружениями} {\it 229}
\item {\texttt{square-of-four}} {\it 139}
\item {\texttt{squarer} (ограничение)} {\it 280} (упр.~3.34), {\it 280} (упр.~3.35)
\item {\texttt{squash-inwards}} {\it 144}
\item {\texttt{stack-inst-reg-name}} {\it 483}
\item {\texttt{start}} 472, {\it 475}
\item {\texttt{start-eceval}} {\it 552}{\it п}
\item {\texttt{start-segment}} 99 (упр.~2.2), 142 (упр.~2.48)
\item {\texttt{statements}} {\it 536}
\item {\texttt{stream-append}} {\it 319}
\item {\texttt{stream-append-delayed}} {\it 442}
\item {\texttt{stream-car}} 301, {\it 302}
\item {\texttt{stream-cdr}} 301, {\it 302}
\item {\texttt{stream-enumerate-interval}} {\it 303}
\item {\texttt{stream-filter}} {\it 303}
\item {\texttt{stream-flatmap}} {\it 443}, 447 (упр.~4.74)
\item {\texttt{stream-for-each}} {\it 301}
\item {\texttt{stream-limit}} 317 (упр.~3.64)
\item {\texttt{stream-map}} {\it 301}
  \subitem {с несколькими аргументами} {\it 306} (упр.~3.50)
\item {\texttt{stream-null?}} 301
  \subitem {в MIT Scheme} 301{\it п}
\item {\texttt{stream-ref}} {\it 301}
\item {\texttt{stream-withdraw}} {\it 332}
\item {\texttt{sub} (обобщенная)} {\it 187}
\item {\texttt{sub-complex}} {\it 173}
\item {\texttt{sub-interval}} 103 (упр.~2.8)
\item {\texttt{sub-rat}} {\it 94}
\item {\texttt{sub-vect}} 141 (упр.~2.46)
\item {\texttt{subsets}} {\it 120} (упр.~2.32)
\item {\texttt{sum}} {\it 71}
  \subitem {итеративный вариант} 73 (упр.~1.30)
  \subitem {как накопление} 74 (упр.~1.32)
\item {\texttt{sum-cubes}} {\it 70}
  \subitem {через процедуры высших порядков} {\it 71}
\item {\texttt{sum-integers}} {\it 70}
  \subitem {через процедуры высших порядков} {\it 72}
\item {\texttt{sum-odd-squares}} {\it 120}, {\it 123}
\item {\texttt{sum-of-squares}} {\it 32}
  \subitem {в модели с окружениями} 231
\item {\texttt{sum-primes}} {\it 300}
\item {\texttt{sum?}} 150, {\it 152}
\item {\texttt{symbol-leaf}} {\it 166}
\item {\texttt{symbol?} (элементарная процедура)} {\it 151}
  \subitem {и тип данных} 191 (упр.~2.78)
  \subitem {реализация через типизированные указатели} 493
\item {\texttt{symbols}} {\it 167}
\item {SYNC} 298{\it п}
\bigskip
\item {\texttt{\#t}} 36{\it п}
\item {\texttt{tack-on-instruction-sequence}} {\it 538}
\item {\texttt{tagged-list?}} {\it 344}
\item {\texttt{term-list}} {\it 201}
\item {\texttt{test} (в регистровой машине)} 454
  \subitem {имитация} 482
\item {\texttt{test-and-set!}} {\it 295}, {\it 296}{\it п}
\item {\texttt{test-condition}} {\it 482}
\item {\texttt{text-of-quotation}} {\it 344}
\item {\texttt{\texttt{lambda}-выражение}}
  \subitem {как оператор~в комбинации} 75
\item {THE, Система Мультипрограммирования} 294{\it п}
\item {\texttt{the-cars}}
  \subitem {вектор} 490
  \subitem {регистр} 492, 496
\item {\texttt{the-cdrs}}
  \subitem {вектор} 490
  \subitem {регистр} 492, 496
\item {\texttt{the-empty-environment}} {\it 352}
\item {\texttt{the-empty-stream}} 301
  \subitem {в MIT Scheme} 301{\it п}
\item {\texttt{the-empty-termlist}} 201, {\it 204}
\item {\texttt{the-global-environment}} {\it 355}, {\it 512}{\it п}
\item {\texttt{thunk-env}} {\it 375}
\item {\texttt{thunk-exp}} {\it 375}
\item {\texttt{thunk?}} {\it 375}
\item {\texttt{timed-prime-test}} {\it 67} (упр.~1.22)
\item {TK!Solver} 272{\it п}
\item {\texttt{transform-painter}} {\it 143}
\item {\texttt{transpose}} 127 (упр.~2.37)
\item {\texttt{tree->list...}} {\it 161} (упр.~2.63)
\item {\texttt{tree-map}} 119 (упр.~2.31)
\item {\texttt{true}} {\it 36}{\it п}
\item {\texttt{true?}} {\it 350}
\item {\texttt{try-again}} {\it 385}
\item {\texttt{type-tag}} {\it 176}
  \subitem {использование типов Scheme} 191 (упр.~2.78)
\bigskip
\item {\texttt{unev}, регистр} 501
\item {\texttt{unify-match}} {\it 437}
\item {\texttt{union-set}} 154
  \subitem {представление в виде бинарных деревьев} 162 (упр.~2.65)
  \subitem {представление в виде неупорядоченных списков} 156 (упр.~2.59)
  \subitem {представление в виде упорядоченных списков} 158 (упр.~2.62)
\item {\texttt{unique} (язык запросов)} 447 (упр.~4.75)
\item {\texttt{unique-pairs}} 130 (упр.~2.40)
\item {UNIX (Юникс)} 516{\it п}, 554{\it п}
\item {\texttt{unknown-expression-type}} {\it 513}
\item {\texttt{unknown-procedure-type}} {\it 513}
\item {\texttt{up-split}} 137 (упр.~2.44)
\item {\texttt{update-insts!}} {\it 479}
\item {\texttt{upper-bound}} 103 (упр.~2.7)
\item {\texttt{user-initial-environment}} {\it 360}{\it п}
\item {\texttt{user-print}} {\it 356}
  \subitem {измененная для скомпилированного кода} {\it 552}{\it п}
\bigskip
\item {V-операция на семафоре} 294{\it п}
\item {\texttt{val}, регистр} 501
\item {\texttt{value-proc}} 481
\item {\texttt{variable}} {\it 201}
\item {\texttt{variable?}} 150, {\it 343}
\item {\texttt{vector-ref} (элементарная процедура)} {\it 490}
\item {\texttt{vector-set!} (элементарная процедура)} {\it 490}
\item {\texttt{verbs}} {\it 389}
\bigskip
\item {\texttt{weight}} {\it 167}
\item {\texttt{weight-leaf}} {\it 166}
\item {\texttt{width}} {\it 104}
\item {\texttt{withdraw}} {\it 213}
  \subitem {сложности в параллельных системах} {\it 284}
\item {\texttt{without-interrupts}} 296{\it п}
\bigskip
\item {\texttt{xcor-vect}} 141 (упр.~2.46)
\item {Xerox, исследовательский центр~в Пало Альто [Xerox Palo Alto Research Center]} 24{\it п}
\bigskip
\item {Y-оператор [Y operator]} 364{\it п}
\item {\texttt{ycor-vect}} 141 (упр.~2.46)
\bigskip
\item {Zetalisp (Зеталисп)} 24{\it п}
\bigskip
\item {Абельсон, Харольд [Harold Abelson]} 24{\it п}
\item {абстрактные данные [abstract data]} 93, {\it см. также} абстракция данных
\item {абстрактные модели данных [abstract models for data]} 100{\it п}
\item {абстрактный синтаксис [abstract syntax]}
  \subitem {в метациклическом интерпретаторе} 339
  \subitem {в языке запросов} 430
\item {абстракция [abstraction]} {\it см. также} средства абстракции; абстракция данных; процедуры высших порядков
  \subitem {выделение общей схемы} 71
  \subitem {метаязыковая} 336
  \subitem {поиска в недетерминистском программировании} 387
  \subitem {при проектировании регистровых машин} 457
  \subitem {процедурная} 44
\item {абстракция данных [data abstraction]} 91, 93, 170, 173, 343, {\it см. также} метациклический интерпретатор
  \subitem {для очереди} 250
\item {автомагически [automagically]} 384
\item {автоматический поиск [automatic search]} 381, {\it см.~также} поиск
  \subitem {история} 384{\it п}
\item {автоматическое распределение памяти [automatic storage allocation]} 489
\item {Ада, сыновья} 417 (упр.~4.63)
\item {Адамс, Норман~И.,~IV [Norman I. Adams IV]} 366{\it п}
\item {аддитивность [additivity]} 93, 171, 179
\item {Адельман, Леонард [Leonard Adleman]} 67{\it п}
\item {адрес [address]} 489
\item {адресная арифметика [address arithmetic]} 490
\item {азбука Морзе [Morse code]} 163
\item {Аккермана функция [Ackermann's function]} 52 (упр.~1.10)
\item {алгебра символьная} {\it см.} символьная алгебра
\item {алгебраическая спецификация [algebraic specification]} 100{\it п}
\item {алгебраическое выражение [algebraic expression]} 198
  \subitem {дифференцирование} 149
  \subitem {представление} 151
  \subitem {упрощение} 152
\item {алгоритм [algorithm]}
  \subitem {RSA [RSA]} 67{\it п}
  \subitem {вероятностный} 66
  \subitem {Евклида [Euclid's]} 63, 451
  \subitem {оптимальный} 125{\it п}
  \subitem {унификации [unification algorithm]} 405{\it п}
\item {Аллен, Джон [John Allen]} 495{\it п}
\item {альтернатива \texttt{if} [alternative of {\tt if}]} 37
\item {анализирующий интерпретатор [analyzing evaluator]} 365
  \subitem {\texttt{let}} 369 (упр.~4.22)
  \subitem {как основа для недетерминистского интерпретатора} 394
\item {Аппель, Эндрю~У. [Andrew~W. Appel]} 535{\it п}
\item {аппликативный порядок вычислений [applicative-order evaluation]} 35
  \subitem {vs. нормальный порядок} 39 (упр.~1.5), 64 (упр.~1.20), 370
  \subitem {в Лиспе} 35
\item {арбитр [arbiter]} 296{\it п}
\item {аргумент(ы) [argument(s)]} 26
  \subitem {задержанный} 325
  \subitem {произвольное количество} 26, 112 (упр.~2.20)
\item {Аристотель, {\em De caelo} (комментарий Ж.~Буридана)} 296{\it п}
\item {арифметика [arithmetic]}
  \subitem {интервальная} 102
  \subitem {комплексных чисел} 171, {\it см.} арифметика комплексных чисел
  \subitem {многочленов} {\it см.} арифметика многочленов
  \subitem {обобщенные операции} 186
  \subitem {рациональная} 93
  \subitem {элементарные процедуры} 26
\item {арифметика комплексных чисел [complex-number arithmetic]} 171
  \subitem {взаимодействие c общими арифметическими системами} 189
  \subitem {структура системы} 178
\item {арифметика многочленов [polynomial arithmetic]} 199
  \subitem {алгоритм Евклида} 207{\it п}
  \subitem {вероятностный алгоритм для НОД} 210{\it п}
  \subitem {включение в систему обобщенной арифметики} 200
  \subitem {вычитание} 205 (упр.~2.88)
  \subitem {деление} 205 (упр.~2.91)
  \subitem {наибольший общий делитель} 207, 210{\it п}
  \subitem {рациональные функции} 207
  \subitem {сложение} 200
  \subitem {умножение} 200
\item {арктангенс [arctangent]} 174{\it п}
\item {ассемблер [assembler]} 473, 476
\item {атомарность [atomicity]} 295
\item {атомарные операции, поддерживаемые на уровне аппаратуры [atomic operations supported in hardware]} 296
\item {А'х-мосе [A'h-mose]} 61{\it п}
\item {Ачарья, Бхаскара [Bh\'ascara \'Ach\'arya]} 57{\it п}
\bigskip
\item {база данных [data base]}
  \subitem {и логическое программирование} 407
  \subitem {и программирование, управляемое данными} 184 (упр.~2.74)
  \subitem {индексирование} 418{\it п}, 439
  \subitem {как множество записей} 162
  \subitem {персонал компании Insatiable Enterprises, Inc.} 184 (упр.~2.74)
  \subitem {персонал <<Микрошафт>>} 407
\item {базовый адрес [base address]} 490
\item {банковский счет [bank account]} 213, 240 (упр.~3.11)
  \subitem {защищенный паролем} 217 (упр.~3.3)
  \subitem {мена балансов местами} 291
  \subitem {перенос денег} 293 (упр.~3.44)
  \subitem {потоковая модель} 332
  \subitem {сериализованный} 289
  \subitem {совместный} 225, 226 (упр.~3.7), 284
  \subitem {совместный, смоделированный с помощью потоков} 334
\item {Барт,~ Джон [John Barth]} 335
\item {барьерная синхронизация [barrier synchronization]} 298{\it п}
\item {барьеры абстракции [abstraction barriers]} 92, 97, 170
  \subitem {в обобщенной арифметической системе} 186
  \subitem {в системе работы с комплексными числами} 171
\item {Батали, Джон Дин [John Dean Batali]} 500{\it п}
\item {башня типов [tower of types]} 194
\item {Бейкер, Генри~Дж., мл. [Henry~J. Baker Jr.]} 495{\it п}
\item {Бертрана гипотеза [Bertrand's hypothesis]} 311{\it п}
\item {бесконечная последовательность [infinite series]} 438{\it п}
\item {бесконечные потоки [infinite streams]} 307
  \subitem {простых чисел} {\it см.} \texttt{primes}
  \subitem {целых чисел} {\it см.} \texttt{integers}
  \subitem {чисел Фибоначчи} {\it см.} \texttt{fibs}
\item {бесконечный поток}
  \subitem {для моделирования сигналов} 321
  \subitem {для суммирования ряда} 315
  \subitem {пар} 318
  \subitem {представление степенных рядов} 312 (упр.~3.59)
  \subitem {слияние} 311 (упр.~3.56), 319, 320 (упр.~3.70), 334
  \subitem {слияние как отношение} 334{\it п}
  \subitem {случайных чисел} 330
  \subitem {факториалов} 311 (упр.~3.54)
\item {бинарное дерево [binary tree]} 158
  \subitem {для кодирования по Хаффману} 164
  \subitem {представленное при помощи списков} 159
  \subitem {преобразование в список} 161 (упр.~2.63)
  \subitem {преобразование из списка} 161 (упр.~2.64)
  \subitem {списки, представленные как} 158
  \subitem {таблицы} 260 (упр.~3.26)
\item {бинарный поиск [binary search]} 158
\item {биномиальный коэффициент [binomial coefficients]} 57{\it п}
\item {блоха [bug]} 23
\item {блочная структура [block structure]} 47, 360
  \subitem {в модели окружения} 238
  \subitem {в языке запросов} 449 (упр.~4.79)
\item {<<Болт, Беранек~и Ньюман>> [Bolt, Beranek and Newman, inc.]} 24{\it п}
\item {большое число [bignum]} 491
\item {Борнинг, Алан [Alan Borning]} 272{\it п}
\item {Бородин, Алан [Alan Borodin]} 126{\it п}
\item {Буридан, Жан [Jean Buridan]} 296{\it п}
\item {буфер [buffer]}
  \subitem {FIFO} 250
  \subitem {LIFO} {\it см.} стек
\item {Бытие} 417 (упр.~4.63)
\item {Бэкус, Джон [John Backus]} 333{\it п}
\item {бюрократия [bureaucracy]} 425
\bigskip
\item {Вагнер, Эрик~Дж. [Eric~G. Wagner]} 100{\it п}
\item {Ванд, Митчелл [Mitchell Wand]} 337{\it п}, 505{\it п}
\item {Вейль, Герман [Hermann Weyl]} 90
\item {вектор (математический) [vector (mathematical)]}
  \subitem {в кадре из языка описания изображений} 140
  \subitem {операции} 126 (упр.~2.37), 141 (упр.~2.46)
  \subitem {представленный в виде пары} 141 (упр.~2.46)
  \subitem {представленный в виде последовательности} 126 (упр.~2.37)
\item {вектор (структура данных) [vector (data structure)]} 490
\item {Венера [Venus]} 147{\it п}
\item {вероятностный алгоритм [probabilistic algorithm]} 66, 67, 210{\it п}, 308{\it п}
\item {вершина дерева [node of a tree]} 29
\item {ветвь [clause]} 36
\item {ветвь {\tt cond} [clause, of a {\tt cond}]} 36
  \subitem {дополнительный синтаксис} 349 (упр.~4.5)
\item {ветвь дерева [branch of a tree]} 29
\item {вечерняя звезда [evening star]} {\it см.} Венера
\item {взаимное исключение [mutual exclusion]} 294{\it п}
\item {взгляд на вычисления как на обработку сигналов [signal-processing view of computation]} 121
\item {Виноград, Терри [Terry Winograd]} 385{\it п}
\item {вложение комбинаций [nested combinations]} 27
\item {вложенные определения [nested definitions]} {\it см.} внутренние определения
\item {вложенные отображения [nested mappings]} {\it см.} отображение
\item {вложенные применения \texttt{car} и \texttt{cdr} [nested applications of {\tt car} and {\tt cdr}]} 108{\it п}
\item {внутреннее состояние [local state]} 212
  \subitem {поддерживаемое в кадрах} 234
\item {внутренние определения [internal definitions]} 46
  \subitem {vs. \texttt{let}} 78
  \subitem {в модели с окружениями} 238
  \subitem {в недетерминистском интерпретаторе} 399{\it п}
  \subitem {ограничения} 361{\it п}
  \subitem {позиция} 47{\it п}
  \subitem {прочесывание и уничтожение} 361
  \subitem {свободная переменная внутри} 47
  \subitem {сфера действия имени} 360
\item {внутренний язык машины [native language of machine]} 517
\item {внутренняя переменная состояния [local state variable]} 212, 213
\item {возведение~в степень [exponentiation]} 59
  \subitem {по модулю $n$} 65
\item {возврат [backtracking]} 384, {\it см.~также} недетерминисткие вычисления
\item {возврат нескольких значений [returning multiple values]} 478{\it п}
\item {волшебник [wizard]} {\it см.} специалист по численному анализу
\item {вопросительный знак,~в именах предикатов [question mark]} 41{\it п}
\item {восклицательный знак, в именах процедур [exclamation point]} 214{\it п}
\item {восприятие [interning]} 492
\item {временн\'ая диаграмма [timing diargam]} 283
\item {временной отрезок [time segment]} 269
\item {время [time]} 281
  \subitem {в недетерминистских вычислениях} 382
  \subitem {в недетерминистском вычислении} 384
  \subitem {в параллельных системах} 284
  \subitem {и взаимодействие процессов} 298
  \subitem {и присваивание} 281
  \subitem {и функциональное программирование} 331
  \subitem {предназначение} 284{\it п}
\item {встроенный язык, использование в разработке языков [embedded language]} 369
\item {выборки ключа, процедура [key selector]} 162
\item {вывод типов [type inference]} 329{\it п}
\item {выделение стека и хвостовая рекурсия [stack allocation and tail recursion]} 535{\it п}
\item {вызов по имени [call by name]} 372{\it п}
\item {вызов по необходимости [call by need]} 306{\it п}, 372{\it п}
  \subitem {связь с мемоизацией} 312{\it п}
\item {вынуждение санка [forcing a thunk]} 372
\item {выражение [expression]} 26, {\it см. также} составные выражения; элементарные выражения
  \subitem {алгебраическое} {\it см.} алгебраическое выражение
  \subitem {самовычисляющееся} 340
  \subitem {символьное} 92, {\it см.также} символ(ы)
\item {выражение-следствие [consequent expression]}
  \subitem {{\tt cond}} 36
  \subitem {{\tt if}} 37
\item {высокоуровневый язык vs. машинный язык} 336
\item {выход из тупика [deadlock recovery]} 297{\it п}
\item {вычисление [evaluation]} 26
  \subitem {{\tt and}} 37
  \subitem {{\tt cond}} 36
  \subitem {{\tt if}} 37
  \subitem {{\tt or}} 37
  \subitem {аппликативный порядок} {\it см.} аппликативный порядок вычислений
  \subitem {задержанное} {\it см.} задержанные вычисления
  \subitem {комбинации} 29
  \subitem {модели} 354
  \subitem {модель с окружениями} {\it см.} модель вычисления с окружениями
  \subitem {нормальный порядок} {\it см.} нормальный порядок вычислений 
  \subitem {особых форм} 31
  \subitem {подстановочная модель} {\it см.} подстановочная модель применения процедуры
  \subitem {порядок вычисления подвыражений} {\it см.} порядок вычисления
  \subitem {элементарных выражений} 30
\item {вычислимость [computability]} 359{\it п}, 360{\it п}
\item {вычислители} {\it см.} метациклический интерпретатор; анализирующий интерпретатор; недетерминистский интерпретатор; ленивый интерпретатор; интерпретатор запросов; вычислитель с явным управлением
\item {вычислитель [evaluator]} 336, {\it см. также} интерпретатор
  \subitem {как универсальная машина} 359
  \subitem {метациклический} 338
\item {вычислитель c хвостовой рекурсией [tail-recursive evaluator]} 508
\item {вычислитель с нормальным порядком} {\it см.} ленивый интерпретатор
\item {вычислитель с явным управлением для Scheme [explicit-control evaluator for Scheme]} 500
  \subitem {выражения без подвыражений, подлежащих вычислению} 502
  \subitem {вычисление операндов} 504
  \subitem {запуск} 512
  \subitem {использование стека} 503
  \subitem {как программа на машинном языке} 517
  \subitem {как универсальная машина} 517
  \subitem {комбинации} 503
  \subitem {контроллер} 502
  \subitem {модель машины} 513
  \subitem {модифицированный для скомпилированного кода} 551
  \subitem {нормальный порядок вычислений} 511 (упр.~5.25)
  \subitem {обработка ошибок} 512, 516 (упр.~5.30)
  \subitem {операции} 501
  \subitem {определения} 511
  \subitem {оптимизация} 524 (упр.~5.32)
  \subitem {отслеживание производительности (использование стека)} 514
  \subitem {последовательности выражений} 507
  \subitem {применение процедур} 503
  \subitem {присваивания} 510
  \subitem {производные выражения} 511 (упр.~5.23), 511 (упр.~5.24)
  \subitem {пути данных} 501
  \subitem {регистры} 501
  \subitem {составная процедура} 506
  \subitem {управляющий цикл} 512
  \subitem {условные выражения} 510
  \subitem {хвостовая рекурсия} 508, 515 (упр.~5.26), 516 (упр.~5.28)
  \subitem {элементарные процедуры} 506
\item {вычислительный процесс [computational process]} 22, {\it см. также} процесс
\bigskip
\item {Гаттэг, Джон Фогель [John Vogel Guttag]} 100{\it п}
\item {генератор кода [code generator]} 520
  \subitem {аргументы} 520
  \subitem {возвращаемое значение} 521
\item {генератор случайных чисел [random-number generator]} 213{\it п}, 218
  \subitem {в моделировании методом Монте-Карло} 219
  \subitem {в тесте на простоту} 65
  \subitem {со сбрасыванием} 221 (упр.~3.6)
  \subitem {со сбрасыванием, потоковая версия} 331 (упр.~3.81)
\item {Гераклит} 211
\item {Герон Александрийский} 40{\it п}
\item {гипотеза о замкнутости мира [closed world assumption]} 428
\item {глобальное окружение [global environment]} 28, 228
  \subitem {в метациклическом интерпретаторе} 354
\item {глобальное поведение процесса [global behavior of a process]} 48
\item {глобальный кадр [global frame]} 227
\item {глюк [glitch]} 23
\item {Гоген, Джозеф [Joseph Goguen]} 100{\it п}
\item {Гордон, Майкл [Michael Gordon]} 329{\it п}
\item {Горнер,~У.Дж. [W.~J.~Horner]} 125 (упр.~2.34)
\item {грамматика [grammar]} 389
\item {графика [graphics]} {\it см.} язык описания изображений
\item {Грей, Джим [Jim Gray]} 297{\it п}
\item {Грин, Корделл [Cordell Green]} 405{\it п}
\item {Грисс, Мартин Льюис [Martin Lewis Griss]} 24{\it п}
\item {Гэбриел, Ричард~П. [Richard~P. Gabriel]} 364{\it п}
\bigskip
\item {Дайнсман, Говард~П. [Howard~P. Dinesman]} 387
\item {данные [data]} 22, 25, 100
  \subitem {абстрактные} 93, {\it см.~также} абстракция данных
  \subitem {абстрактные модели} 100{\it п}
  \subitem {алгебраическая спецификация} 100{\it п}
  \subitem {значение} 100
  \subitem {иерархические} 106, 115
  \subitem {изменяемые} {\it см.} изменяемые объекты данных
  \subitem {как программы} 357
  \subitem {<<конкретное представление>>} 93
  \subitem {помеченные} 175, 490{\it п}
  \subitem {процедурное представление} 100
  \subitem {разделенные} 246
  \subitem {символьные} 146
  \subitem {со списковой структурой [list-structured]} 96
  \subitem {составные} 90
  \subitem {численные} 25
\item {Дарлингтон, Джон [John Darlington]} 333{\it п}
\item {двоичные числа, сложение [binary numbers, addition of]} {\it см.} сумматор
\item {де Клеер, Йохан [Johan deKleer]} 385{\it п}, 427{\it п}
\item {Дейкстра, Эдсгер Вибе [Edsger Wybe Dijkstra]} 294{\it п}
\item {действия,~в регистровой машине [actions, in register machine]} 457
\item {дек [deque]} 254 (упр.~3.23)
\item {декларативное vs. императивное знание [declarative vs. imperative knowledge]} 40, 404
  \subitem {и логическое программирование} 405, 425
  \subitem {и недетерминистское вычисление} 382{\it п}
\item {декомпозиция программы [decomposition of program into parts]} 44
\item {деление целых чисел [division of integers]} 42{\it п}
\item {деньги, размен} {\it см.} размен денег
\item {дерево [tree]} 115
  \subitem {B-дерево} 160{\it п}
  \subitem {бинарное} 158, {\it см. также} бинарное дерево
  \subitem {красно-черное дерево} 160{\it п}
  \subitem {ленивое} 380{\it п}
  \subitem {листва} 118 (упр.~2.28)
  \subitem {обращение на всех уровнях} 117 (упр.~2.27)
  \subitem {отображение} 119
  \subitem {перечисление листьев} 123
  \subitem {подсчет числа листьев} 115
  \subitem {представление комбинации} 29
  \subitem {представленное в виде пар} 115
  \subitem {Хаффмана} 164
\item {десятичная точка~в числах [decimal point in numbers]} 42{\it п}
\item {Джаяраман, Сундаресан [Sundaresan Jayaraman]} 272{\it п}
\item {диаграмма потока сигналов [signal-flow diagram]} 121, 322 (упр.~3.73)
\item {диалекты Лиспа [Lisp dialects]}
  \subitem {Common Lisp} 24{\it п}
  \subitem {Franz Lisp (Франц Лисп)} 24{\it п}
  \subitem {InterLisp (ИнтерЛисп)} 24{\it п}
  \subitem {MacLisp (МакЛисп)} 24{\it п}
  \subitem {MDL} 496{\it п}
  \subitem {Portable Standard Lisp (Переносимый Стандартный Лисп)} 24{\it п}
  \subitem {Scheme (Схема)} 24
  \subitem {Zetalisp (Зеталисп)} 24{\it п}
\item {Диофант, \emph{Арифметика}; экземпляр Ферма} 65{\it п}
\item {диспетчеризация [dispatching]}
  \subitem {по типу [on type]} 179
  \subitem {сравнение различных стилей} 186 (упр.~2.76)
\item {диспетчирование}
  \subitem {по типу} {\it см.~также} программирование, управляемое данными
\item {дисциплина кадрированного стека [framed-stack discipline]} 503{\it п}
\item {дифференциальное уравнение [differential equation]} 324, {\it см.~также} \texttt{solve}
  \subitem {второго порядка} 327 (упр.~ 3.78), 328 (упр.~3.79)
\item {дифференцирование [differentiation]}
  \subitem {правила} 150, 153 (упр.~2.56)
  \subitem {символьное} 149, 183 (упр.~2.73)
  \subitem {численное} 85
\item {диффузия, имитация [imitation of diffusion]} 287
\item {Дойл, Джон [Jon Doyle]} 385{\it п}
\item {доказательство корректности программы [proving programs correct]} 40{\it п}
\item {доказательство теорем (автоматическое) [automatic theorem proving]} 405{\it п}
\item {древовидная рекурсия [tree recursion]} 53
  \subitem {порядок роста} 58
\item {древовидно-рекурсивное вычисление чисел Фибоначчи [tree-recursive Fi\-bo\-nac\-ci-number computation]} 53
\item {дробь [fraction]} {\it см.} рациональные числа
\bigskip
\item {Евклид, \emph{Начала}} 63{\it п}
\item {Евклида алгоритм [Euclid's Algorithm]} 63, 451, {\it см. также} наибольший общий делитель
  \subitem {для многочленов} 207{\it п}
  \subitem {порядок роста} 63
\item {единичный квадрат [unit square]} 140
\item {естественный язык [natural language]}
  \subitem {кавычки} 146
  \subitem {синтаксический анализ} {\it см.} синтаксический анализ естественного языка
\bigskip
\item {\texttt{живет-около} (правило)} {\it 413}, 415 (упр.~4.60)
\bigskip
\item {Заби, Рамин [Ramin Zabih]} 385{\it п}
\item {заблокированный процесс [blocked process]} 295{\it п}
\item {зависимость от реализации} {\it см. также} неопределенные значения
  \subitem {порядок вычисления подвыражений} 229{\it п}
  \subitem {числа} 42{\it п}
\item {загадки}
  \subitem {задача о восьми ферзях} 130 (упр.~2.42), 389 (упр.~4.44)
  \subitem {логические} 387
\item {задача о восьми ферзях} 130 (упр.~2.42), 389 (упр.~4.44)
\item {задержанные вычисления [delayed evaluation]} 212, 299
  \subitem {в ленивом интерпретаторе} 369
  \subitem {и нормальный порядок вычислений} 328
  \subitem {и печать} 306 (упр.~3.51)
  \subitem {и потоки} 324
  \subitem {и присваивание} 306 (упр.~3.52)
  \subitem {явные и автоматические} 380
\item {задержанный аргумент [delayed argument]} 325
\item {задержанный объект [delayed object]} 302
\item {задержка, в цифровой схеме [delay, in digital circuit]} 261
\item {замкнутости мира гипотеза} {\it см.} гипотеза о замкнутости мира
\item {замыкание [closure]} 92
  \subitem {в абстрактной алгебре} 106{\it п}
  \subitem {отсутствие во многих языках} 107{\it п}
  \subitem {свойство замыкания \texttt{cons}} 106
  \subitem {свойство замыкания языка описания картинок} 132, 134
\item {замыкания свойство [closure property]} 106
\item {занятое ожидание [busy waiting]} 295{\it п}
\item {запись, в базе данных [record, in a data base]} 162
\item {запрос [query]} 406, {\it см. также} простой запрос; составной запрос
  \subitem {правило} {\it см. также} правило (в языке запросов)
  \subitem {простой} {\it см.} простой запрос
  \subitem {составной} {\it см.} составной запрос
\item {запятая внутри обратной кавычки [comma, used with backquote]} 524{\it п}
\item {зарезервированные слова [reserved words]} 547 (упр.~5.38), 550 (упр.~5.44)
\item {захват мьютекса [acquiring a mutex]} 294
\item {захват свободной переменной [free variable capture]} 46
\item {защищенный паролем банковский счет [password-protected bank account]} 217 (упр.~3.3)
\item {Земля, измерение длины окружности [measuring circumference of Earth]} 308{\it п}
\item {Зиллес, Стивен~Н. [Stephen~N. Zilles]} 100{\it п}
\item {Зиппель, Ричард~Э. [Richard~E. Zippel]} 210{\it п}
\item {значение [value]}
  \subitem {выражения} 27{\it п}, {\it см. также} неопределенные значения
  \subitem {комбинации} 26
  \subitem {переменной} 28
\item {золотое сечение [golden ratio]} 53
  \subitem {как неподвижная точка} 82 (упр.~1.35)
  \subitem {как цепная дробь} 82 (упр.~1.37)
\bigskip
\item {И-элемент [and-gate]} 261
\item {иерархические структуры данных} 106, 115
\item {иерархия типов [hierarchy of types]} 194
  \subitem {в символьной алгебре} 206
  \subitem {неадекватность} 195
\item {извлечение информации [information retrieval]} {\it см.} база данных
\item {изменение и тождественность [change and sameness]}
  \subitem {значение} 223
  \subitem {и разделяемые данные} 246
\item {изменяемые объекты данных} 241, {\it см. также} очередь; таблица
  \subitem {пары} 242
  \subitem {процедурное представление} 249
  \subitem {разделяемые данные} 248
  \subitem {реализованные с помощью присваиваний} 249
  \subitem {списковая структура} 242
\item {изменяемые регистры [modified registers]} {\it см.} последовательность команд
\item {ИЛИ-элемент [or-gate]} 261
\item {именование [naming]}
  \subitem {вычислительных объектов} 27
  \subitem {процедур} 32
\item {именованный \texttt{let} (особая форма) [named {\tt let}]} {\it 349} (упр.~4.8)
\item {имитатор регистровых машин [register-machine simulator]} 472
\item {имитация [simulation]}
  \subitem {для отслеживания производительности регистровой машины} 486
  \subitem {как инструмент для проектирования машин} 514
  \subitem {регистровых машин} {\it см.} имитатор регистровых машин
  \subitem {управляемая событиями [event-driven]} 261
  \subitem {цифровых схем} {\it см.} имитация цифровых схем
\item {имитация цифровых схем [digital-circuit simulation]} 260
  \subitem {план действий} 267
  \subitem {представление проводов} 265
  \subitem {пример работы модели} 268
  \subitem {реализация плана действий} 269
  \subitem {элементарные функциональные элементы} 263
\item {императивное vs. декларативное знание [imperative vs. declarative knowledge]} 40, 404
  \subitem {и логическое программирование} 405, 425
  \subitem {и недетерминистское вычисление} 382{\it п}
\item {императивное программирование [imperative programming]} 225
\item {императивный стиль vs. стиль, ориентированный на выражения [imperative vs. expression-oriented programming style]} 281{\it п}
\item {имя [name]} {\it см. также} локальная переменная; локальное имя; переменная
  \subitem {инкапсулированное} 215{\it п}
  \subitem {процедуры} 32
  \subitem {формального параметра} 45
\item {инвариант итеративного процесса [invariant quantity of an iterative process]} 61 (упр.~1.16)
\item {инвертор [inverter]} 261
\item {Ингерман, Питер [Peter Ingerman]} 372{\it п}
\item {индекс [index]} 490
\item {индексирование базы данных [database indexing]} 418{\it п}, 439
\item {инкапсулированное имя [encapsulated name]} 215{\it п}
\item {инкапсуляция [encapsulation]} 215{\it п}
\item {интеграл [integral]} {\it см. также} опредеденный интеграл; Монте-Карло, интегрирование методом
  \subitem {степенного ряда} 313 (упр.~3.59)
\item {интегратор [integrator]} 322
\item {интерпретатор [interpreter]} 23, 336, {\it см. также} вычислитель
  \subitem {vs. компилятор} 517, 554
  \subitem {цикл чтение-вычисление-печать} 27
\item {интерпретатор \texttt{amb} [{\tt amb} evaluator]} {\it см.} недетерминисткий интерпретатор
\item {интерпретатор языка запросов [query interpreter]} 406
  \subitem {vs. интерпретатор для Лиспа} 423, 424, 448 (упр.~4.79)
  \subitem {база данных} 439
  \subitem {бесконечные циклы} 426, 429 (упр.~4.67)
  \subitem {вычислитель запросов} 424, 431
  \subitem {добавление утверждения или правила} 424
  \subitem {кадр} 227
  \subitem {кадры} 445
  \subitem {конкретизация} 430
  \subitem {обзор} 417
  \subitem {операции над потоками} 442
  \subitem {потоки кадров} 418, 424{\it п}
  \subitem {представление переменной образца} 430
  \subitem {преобразование переменной образца} 444
  \subitem {проблемы с \texttt{not} и \texttt{lisp-value}} 427, 448 (упр.~4.77)
  \subitem {синтаксис языка запросов} 443
  \subitem {сопоставление с образцом} 417, 434
  \subitem {структура окружений} 449 (упр.~4.79)
  \subitem {улучшения} 429 (упр.~4.67), 448 (упр.~4.76), 448 (упр.~4.77)
  \subitem {унификация} 421, 437
  \subitem {управляющий цикл} 424, 429
\item {инфиксная нотация vs. префиксная нотация [infix notation vs. prefix notation]} 154 (упр.~2.58)
\item {информатика [computer science]} 336, 359{\it п}
  \subitem {vs. математика} 40, 404
\item {исполнительная процедура [execution procedure]} 366
  \subitem {в анализирующем вычислителе} 366
  \subitem {в имитаторе регистровых машин} 475, 480
  \subitem {в недетерминистском вычислителе} 394, 396
\item {исполнительная процедура команды [instruction execution procedure]} 475
\item {истина [true]} 36{\it п}
\item {исходная программа [source program]} 517
\item {исходный язык [source language]} 517
\item {итеративные конструкции [iteration costructs]} {\it см.} циклические конструкции
\item {итеративный процесс [iterative process]} 50
  \subitem {vs. рекурсивный процесс} 48, 233 (упр.~3.9), 463, 544 (упр.~5.34)
  \subitem {как потоковый процесс} 314
  \subitem {линейный} 51, 58
  \subitem {построение алгоритмов} 61 (упр.~1.16)
  \subitem {реализованный с помощью вызова процедуры} 509, {\it см. также} хвостовая рекурсия
  \subitem {реализованный~с помощью вызова процедуры} 42, 51
  \subitem {регистровая машина} 463
\bigskip
\item {Йохельсон, Джером~К. [Jerome~C. Yochelson]} 495{\it п}
\bigskip
\item {кавычка [quote]} 146
\item {кавычки} 146
  \subitem {в естественном языке} 146
  \subitem {одинарная vs. двойные} 147{\it п}
  \subitem {с объектами данных Лиспа} 147
  \subitem {со строкой символов} 147{\it п}
\item {кадр (в интерпретаторе запросов) [frame]} 418, {\it см. также} сопоставление с образцом; унификация
  \subitem {представление} 445
\item {кадр (в модели с окружениями) [frame]} 227
  \subitem {глобальный} 227
  \subitem {как хранилище внутреннего состояния} 234
\item {кадрированного стека дисциплина} {\it см.} дисциплина кадрированного стека
\item {``как сделать'' vs. ``что такое'' [``what is'' vs. ``how to'']} {\it см.} императивное vs. декларативное знание
\item {Калдевай, Анне [Anne Kaldewaij]} 62{\it п}
\item {Калифорнийский университет~в Беркли [University of California at Berkeley]} 24{\it п}
\item {калькулятор; поиск неподвижных точек [fixed points with calculator]} 81{\it п}
\item {каноническая форма многочленов [canonical form for polynomials]} 206
\item {Кармайкла числа [Carmichael numbers]} 66{\it п}, 69 (упр.~1.27)
\item {Карр, Альфонс [Alphonse Karr]} 211
\item {каскадный сумматор [ripple-carry adder]} 265 (упр.~3.30)
\item {квадратный корень} 40, {\it см. также} \texttt{sqrt}
  \subitem {поток приближений} 314
\item {квазикавычка [quasiquote]} 524{\it п}
\item {квантовая механика [quantum mechanics]} 334{\it п}
\item {Кеплер, Иоганн [Johannes Kepler]} 450
\item {Кларк, Кит~Л. [Keith~L. Clark]} 428{\it п}
\item {Клингер, Уильям [William Clinger]} 348{\it п}, 372{\it п}
\item {ключ записи [key of a record]}
  \subitem {в базе данных} 162
  \subitem {в таблице} 254
  \subitem {проверка на равенство} 259 (упр.~3.24)
\item {Кнут, Дональд~Э. [Donald~E. Knuth]} 57{\it п}, 61{\it п}, 63{\it п}, 125{\it п}, 218{\it п}, 219{\it п}, 566
\item {Ковальски, Роберт [Robert Kowalski]} 405{\it п}
\item {код [code]}
  \subitem {ASCII} 163
  \subitem {префиксный} 164
  \subitem {с переменной длиной [variable-length code]} 163
  \subitem {с фиксированной длиной [fixed-length code]} 163
  \subitem {Хаффмана} {\it см.} код Хаффмана
\item {код-разделитель [separator code]} 164
\item {Колмогоров,~А.Н.} 218{\it п}
\item {Кольбеккер, Юджин Эдмунд, мл. [Eugene Edmund Kohlbecker Jr.]} 348{\it п}
\item {Кольмероэ, Ален [Alain Colmerauer]} 405{\it п}
\item {кольцо [ring]}
  \subitem {Евклидово [Euclidean]} 207{\it п}
\item {команда [instruction]} 450, 454
\item {комбинация} 26
  \subitem {\texttt{lambda}-выражение как оператор} 75
  \subitem {в виде дерева} 29
  \subitem {вычисление} 29
  \subitem {как оператор комбинации} 84{\it п}
  \subitem {с оператором-комбинацией} 84{\it п}
  \subitem {составное выражение как оператор} 39 (упр.~1.4)
\item {комментарии в программах [comments in programs]} 129{\it п}
\item {компилятор [compiler]} 517
  \subitem {vs. интерпретатор} 517, 554
  \subitem {хвостовая рекурсия, выделение памяти на стеке и сборка мусора} 535{\it п}
\item {компилятор для Scheme} 518, {\it см. также} генератор кода; окружение времени компиляции; последовательность команд; тип связи; целевой регистр
  \subitem {\texttt{lambda}-выражения} 528
  \subitem {vs. анализирующий интерпретатор} 519, 520
  \subitem {vs. вычислитель с явным управлением} 518
  \subitem {генераторы кода} {\it см.} \texttt{compile...}
  \subitem {запуск скомпилированного кода} 551
  \subitem {использование машинных операций} 518{\it п}
  \subitem {использование регистров} 518, 518{\it п}, 535{\it п}
  \subitem {использование стека} 521, 523 (упр.~5.31), 546 (упр.~5.37)
  \subitem {кавычки} 525
  \subitem {комбинации} 530
  \subitem {лексическая адресация} 547
  \subitem {определения} 525
  \subitem {отслеживание производительности (использования стека) скомпилированного кода} 553, 555 (упр.~5.45), 555 (упр.~5.46)
  \subitem {переменные} 525
  \subitem {порождение меток} 527{\it п}
  \subitem {порожденный код, обладающий свойством хвостовой рекурсии} 535
  \subitem {порядок вычисления операндов} 546 (упр.~5.36)
  \subitem {последовательности выражений} 528
  \subitem {применение процедур} 530
  \subitem {пример скомпилированного кода} 539
  \subitem {присваивания} 525
  \subitem {процедуры разбора синтаксиса} 520
  \subitem {самовычисляющиеся выражения} 525
  \subitem {связующий код} 524
  \subitem {связь с вычислителем} 551
  \subitem {структура} 520
  \subitem {уничтожение внутренних определений} 549{\it п}, 550 (упр.~5.43)
  \subitem {условные выражения} 526
  \subitem {эффективность} 518
\item {компиляция [compilation]} 517, {\it см.} компилятор
\item {комплексные числа [complex numbers]}
  \subitem {декартово vs. полярное представление} 172
  \subitem {декартово представление} 174
  \subitem {полярное представление} 174
  \subitem {помеченные данные} 175
\item {композиция функций [composition of functions]} 88 (упр.~1.42)
\item {компьютер общего назначения, как универсальная машина [general-purpose computer as universal machine]} 517
\item {конвейеризация [pipelining]} 284{\it п}
\item {конечная цепная дробь [finite continued fraction]} 82 (упр.~1.37)
\item {конкретизация образца [instantiation of a pattern]} 410
\item {конкретное представление данных [concrete data representation]} 93
\item {Конопасек, Милош [Milos Konopasek]} 272{\it п}
\item {конструктор [constructor]} 93
  \subitem {как барьер абстракции} 98
\item {контроллер для регистровой машины [controller for register machine]} 451
  \subitem {диаграмма} 452
\item {контрольная точка [breakpoint]} 488 (упр.~5.19)
\item {концевая вершина дерева [terminal node of a tree]} 29
\item {корень $n$-й степени как неподвижная точка [$n$th root as fixed point]} 88 (упр.~1.45)
\item {корень четвертой степени как неподвижная точка [fourth root as fixed point]} 88 (упр.~1.45)
\item {Кормен, Томас~Г. [Thomas~H. Cormen]} 160{\it п}
\item {корни уравнения [roots of equation]} {\it см.} Ньютона метод; половинного деления метод
\item {корректность программы [correctness of a program]} 40{\it п}
\item {косинус [cosine]}
  \subitem {неподвижная точка функции} 81
  \subitem {степенной ряд} 312 (упр.~3.59)
\item {космическое излучение [cosmic radiation]} 66{\it п}
\item {красивая печать [pretty printing]} 27
\item {красно-черные деревья [red-black trees]} 160{\it п}
\item {кредитные счета, международные [international credit-card accounts]} 298{\it п}
\item {Кресси, Дэвид [David Cressey]} 496{\it п}
\item {криптография [cryptography]} 67{\it п}
\item {кубический корень [cubic root]}
  \subitem {как неподвижная точка} 84
  \subitem {метод Ньютона} 43 (упр.~1.8)
\bigskip
\item {Лагранжа формула интерполяции [Lagrange interpolation formula]} 200{\it п}
\item {Ламэ, Габриэль [Gabriel Lam\'e]} 63{\it п}
\item {Ландин, Питер [Peter Landin]} 31{\it п}, 306{\it п}
\item {Лапальм, Ги [Guy Lapalme]} 366{\it п}
\item {Лапицкий, Виктор} 335
\item {Лейбниц, барон Готфрид Вильгельм фон [Baron Gottfried Wilhelm von Leibniz]}
  \subitem {доказательство Малой теоремы Ферма} 65{\it п}
  \subitem {ряд для вычисления $\pi$} 70{\it п}, 315
\item {Лейзерсон, Чарльз~Э. [Charles~E. Leiserson]} 160{\it п}
\item {лексическая адресация [lexical addressing]} 354{\it п}, 547
  \subitem {лексический адрес} 548
\item {лексическая сфера действия [lexical scoping]} 47
  \subitem {и структура окружений} 548
\item {лексический адрес [lexical address]} 548
\item {лекция, что на ней делать [something to do during a lecture]} 81{\it п}
\item {ленивая пара [lazy pair]} 379
\item {ленивое вычисление [lazy evaluation]} 370
\item {ленивое дерево [lazy tree]} 380{\it п}
\item {ленивый интерпретатор} 369
\item {ленивый список [lazy list]} 379
\item {Либерман, Генри [Henry Lieberman]} 495{\it п}
\item {линейно итеративный процесс [linear iterative process]} 51
  \subitem {порядок роста} 58
\item {линейно рекурсивный процесс [linear recursive process]} 50
  \subitem {порядок роста} 58
\item {линейный рост [linear growth]} 50, 58
\item {Лисков, Барбара Хьюберман [Barbara Huberman Liskov]} 100{\it п}
\item {логарифм, аппроксимация $\ln 2$ [logarithm, approximating $\ln 2$]} 318 (упр.~3.65)
\item {логарифмический рост [logarithmic growth]} 58, 60, 158{\it п}
\item {логические загадки [logic puzzles]} 387
\item {логический вывод [logical deduction]} 415, 425
\item {логическое И [logical and]} 261
\item {логическое ИЛИ [logical or]} 261
\item {логическое программирование [logic programming]} 337, 404, {\it см. также} интерпретатор запросов; язык запросов
  \subitem {vs. математическая логика} 425
  \subitem {в Японии} 406{\it п}
  \subitem {история} 405{\it п}, 406{\it п}
  \subitem {компьютеры} 406{\it п}
  \subitem {язык логического программирования} 406
\item {ложь [false]} 36{\it п}
\item {локальная переменная [local variable]} 76
\item {локальная эволюция процесса [local evolution of a process]} 48
\item {локальное имя} 45, 76
\item {Локк, Джон [John Locke]} 22
\item {Лэмпорт, Лесли [Leslie Lamport]} 298{\it п}
\item {Лэмпсон, Батлер [Butler Lampson]} 225{\it п}
\item {лямбда-исчисление ($\lambda$-исчисление) [$\lambda$ calculus]} 75{\it п}
\bigskip
\item {Макаллестер, Дэвид Аллен [David Allen McAllester]} 385{\it п}
\item {Макдермот, Дрю [Drew McDermott]} 385{\it п}
\item {Маккарти, Джон [John McCarthy]} 23, 23{\it п}, 383{\it п}
\item {макрос [macro]} 348{\it п}, {\it см. также} макросимволы ввода
\item {макросимволы ввода [reader macro characters]} 444{\it п}
\item {Марсельский университет [University of Marseille]} 405{\it п}
\item {математика [mathematics]}
  \subitem {vs. информатика} 40, 404
  \subitem {vs. техника} 66{\it п}
\item {математическая функция} {\it см.} функция (математическая)
\item {матрица, представленная как последовательность [matrix, represented as sequ\-ence]} 126 (упр.~2.37)
\item {Маус, Минни и Микки [Mickie and Minnie Mouse]} 426
\item {машина Тьюринга [Turing machine]} 359{\it п}
\item {машинный язык [machine language]} 517
  \subitem {vs. высокоуровневый язык} 336
\item {мемоизация [memoization]} 57{\it п}, 260 (упр.~3.27), 372
  \subitem {и вызов по необходимости} 312{\it п}
  \subitem {и сборка мусора} 375{\it п}
  \subitem {санков} 372
  \subitem {через \texttt{delay}} 305
\item {мера, в Евклидовом кольце [measure in a Euclidean ring]} 207{\it п}
\item {метациклический интерпретатор для Scheme} 338
  \subitem {\texttt{eval} и \texttt{apply}} 339
  \subitem {\texttt{eval}, управляемая данными} 348 (упр.~4.3)
  \subitem {\texttt{true} и \texttt{false}} 354
  \subitem {абстракция данных} 339, 350 (упр.~4.10), 353
  \subitem {анализирующая версия} 365
  \subitem {глобальное окружение} 354
  \subitem {действия над окружениями} 351
  \subitem {задача} 338{\it п}
  \subitem {запуск} 354
  \subitem {и символьное дифференцирование} 343
  \subitem {комбинации (применение процедур)} 348 (упр.~4.2)
  \subitem {компиляция} 556 (упр.~5.50), 557 (упр.~5.52)
  \subitem {модель вычислений с окружениями} 338
  \subitem {наличие хвостовой рекурсии (неизвестно)} 508
  \subitem {особые формы} 348 (упр.~4.4), 349 (упр.~4.5), 349 (упр.~4.6), 349 (упр.~4.7), 349 (упр.~4.8), 350 (упр.~4.9)
  \subitem {особые формы как производные выражения} 346
  \subitem {порядок вычисления операндов} 343 (упр.~4.1)
  \subitem {представление выражений} 339, 343
  \subitem {представление истины и лжи} 350
  \subitem {представление окружений} 352
  \subitem {представление процедур} 351
  \subitem {производные выражения} 346
  \subitem {процедура высшего порядка} 341{\it п}
  \subitem {реализуемый язык vs. язык реализации} 342
  \subitem {синтаксис интерпретируемого языка} 343, 348 (упр.~4.2), 350 (упр.~4.10)
  \subitem {управляющий цикл} 356
  \subitem {цикл \texttt {eval}--\texttt {apply}} 339
  \subitem {элементарные процедуры} 354
  \subitem {эффективность} 365
\item {метаязыковая абстракция [metalinguistic abstraction]} 336
\item {метка [label]} 454
\item {метка типа [type tag]} 171, 175, 491{\it п}
  \subitem {двухуровневая} 190
\item {мечтать не вредно [wishful thinking]} 94, 150
\item {микросхема для Scheme [Scheme chip]} 500, 501
\item {<<Микрошафт>> (Microshaft)} 407
\item {Миллер, Гэри~Л. [Gary~L. Miller]} 69 (упр.~1.28)
\item {Миллер, Джеймс~С. [James~S. Miller]} 535{\it п}
\item {Миллера-Рабина тест [Miller-Rabin test]} 69 (упр.~1.28)
\item {Милнер, Робин [Robin Milner]} 329{\it п}
\item {Минский, Марвин [Marvin Minsky]} 15, 495{\it п}
\item {мировая линия частицы [world line of a particle]} 299{\it п}, 333{\it п}
\item {многочлен(ы) [polynomial(s)]} 199
  \subitem {вычисление по схеме Горнера} 125 (упр.~2.34)
  \subitem {иерархия типов} 206
  \subitem {каноническая форма} 206
  \subitem {от одной переменной} 199
  \subitem {плотный [dense]} 203
  \subitem {разреженный [sparse]} 203
\item {множество [set]} 154
  \subitem {база данных как множество} 162
  \subitem {операции} 154
  \subitem {перестановки} 129
  \subitem {подмножество} 120 (упр.~2.32)
  \subitem {представленное в виде бинарного дерева} 158
  \subitem {представленное в виде неупорядоченного списка} 155
  \subitem {представленное в виде упорядоченного списка} 156
\item {множитель целости [integerizing factor]} 209
\item {мобиль [mobile]} 118 (упр.~2.29)
\item {модели вычисления [models of evaluation]} 512
\item {моделирование [modeling]}
  \subitem {в науке~и технике} 34
  \subitem {как стратегия разработки} 211
\item {моделирование методом Монте-Карло [Monte Carlo simulation]} 219
  \subitem {формулировка в терминах потоков} 330
\item {модель вычисления с окружениями [environment model of evaluation]} 212, 227
  \subitem {внутреннее состояние} 234
  \subitem {внутренние определения} 238
  \subitem {и метациклический интерпретатор} 338
  \subitem {и хвостовая рекурсия} 234{\it п}
  \subitem {передача сообщений} 240 (упр.~3.11)
  \subitem {правила вычисления} 228
  \subitem {пример применения процедуры} 231
  \subitem {структура окружений} 227
\item {модуль [absolute value]} 36
\item {модульность} 124, 211
  \subitem {границы модулей, проведенные по границам объектов} 334{\it п}
  \subitem {и потоки} 314
  \subitem {принцип сокрытия} 215{\it п}
  \subitem {функциональных программ vs. объектов} 330
  \subitem {через бесконечные потоки} 331
  \subitem {через диспетчеризацию по типу} 179
  \subitem {через моделирование объектов} 218
\item {момент времени [moment in time]} 282
\item {Монте-Карло, интегрирование методом [Monte Carlo integration]} 221 (упр.~3.5)
  \subitem {формулировка в терминах потоков} 331 (упр.~3.82)
\item {Морзе азбука} {\it см.} азбука Морзе
\item {Моррис,~Дж.~Г. [J.~H.~Morris]} 225{\it п}
\item {Мун, Дэвид~А. [David~A. Moon]} 24{\it п}, 495{\it п}
\item {Мунро, Иан [Ian Munro]} 126{\it п}
\item {мусор [garbage]} 495
\item {мутатор [mutator]} 241
\item {мьютекс [mutex]} 294
\bigskip
\item {на\-именьших обязательств принцип [principle of least commitment]} 175
\item {надтип [supertype]} 194
  \subitem {несколько надтипов} 195
\item {наибольший общий делитель [greatest common divisor]} 62, {\it см. также} \texttt{gcd}
  \subitem {для многочленов} 207
  \subitem {используемый в арифметике рациональных чисел} 97
  \subitem {используемый для оценки $\pi$} 219
  \subitem {обобщенный} 208 (упр.~2.94)
\item {накопитель [accumulator]} 121, 217 (упр.~3.1)
\item {накопление [accumulation]} 74 (упр.~1.32)
  \subitem {по дереву} 29
\item {невычислимость [noncomputability]} 360{\it п}
\item {недетерминизм в поведении параллельных программ} 287{\it п}, 334{\it п}
\item {недетерминистские вычисления [non\-de\-ter\-mi\-nis\-tic computing]} 337
\item {недетерминистские программы [nondeterministic programs]}
  \subitem {логические загадки} 387
  \subitem {пары чисел с простой суммой} 381
  \subitem {Пифагоровы тройки} 386 (упр.~4.35), 386 (упр.~4.36), 386 (упр.~4.37)
  \subitem {синтаксический анализ естественного языка} 389
\item {недетерминистский интерпретатор [nondeterministic evaluator]} 394
  \subitem {порядок вычисления операндов} 393 (упр.~4.46)
\item {недетерминистское вычисление [non\-de\-ter\-mi\-nis\-tic computing]} 381
\item {недетерминистское программирование vs. программирование на Scheme [non\-deter\-mi\-nis\-tic programming vs. Scheme programming]} 381, 388 (упр.~4.41), 389 (упр.~4.44), 448 (упр.~4.78)
\item {неопределенные значения [unspecified values]}
  \subitem {\texttt{define}} 28{\it п}
  \subitem {\texttt{display}} 96{\it п}
  \subitem {\texttt{if} без альтернативы} 271{\it п}
  \subitem {\texttt{newline}} 96{\it п}
  \subitem {\texttt{set!}} 214{\it п}
  \subitem {\texttt{set-car!}} 242{\it п}
  \subitem {\texttt{set-cdr!}} 242{\it п}
\item {неподвижная точка [fixed point, of a function]} 80
  \subitem {в методе Ньютона} 84
  \subitem {вычисление~с помощью калькулятора} 81{\it п}
  \subitem {золотое сечение} 82 (упр.~1.35)
  \subitem {и унификация} 438{\it п}
  \subitem {как пошаговое улучшение} 89 (упр.~1.46)
  \subitem {квадратный корень} 81, 84, 86
  \subitem {корень $n$-ной степени} 88 (упр.~1.45)
  \subitem {корень четвертой степени} 88 (упр.~1.45)
  \subitem {кубический корень} 84
  \subitem {трансформации некоторой функции} 86
  \subitem {функции косинус} 81
\item {несвязанная переменная [unbound variable]} 227
\item {нестрогая процедура [non-strict procedure]} 371
\item {неудача, в недетерминистских вычислениях}
  \subitem {vs. ошибка} 397
  \subitem {и поиск} 384
\item {неудача, в недетерминистском вычислении [fai\-lure in non\-de\-ter\-mi\-nis\-tic com\-pu\-ta\-tion]} 383
\item {НОД} {\it см.} наибольший общий делитель
\item {номер кадра [frame number]} 548
\item {нормальный порядок вычислений [normal-order evaluation]} 35, 337
  \subitem {\texttt{if}} 39 (упр.~1.5)
  \subitem {vs. аппликативный порядок} 39 (упр.~1.5), 64 (упр.~1.20), 370
  \subitem {в вычислителе с явным управлением} 511 (упр.~5.25)
  \subitem {и задержанные вычисления} 328
\item {нотация в настоящей книге [notation in this book]}
  \subitem {наклонный шрифт для ответов интерпретатора} 26{\it п}
\item {нотация~в настоящей книге}
  \subitem {курсив~в синтаксисе выражений} 32{\it п}
\item {нужные регистры [needed registers]} 523
\item {Ньютона метод [Newton's method]} 84
  \subitem {vs. метод половинного деления} 85{\it п}
  \subitem {для дифференцируемых функций} 84
  \subitem {для квадратных корней} 40, 86
  \subitem {для кубических корней} 43 (упр.~1.8)
\bigskip
\item {область действия переменной [scope of a variable]} 45, {\it см. также} лексическая сфера действия
  \subitem {в \texttt{let}} 77
  \subitem {внутренняя \texttt{define}} 360
  \subitem {формальные параметры процедуры} 45
\item {обмассив [obarray]} 492
\item {обобщенные арифметические операции [generic arithmetic operations]} 187
  \subitem {структура системы} 187
\item {обобщенные операции [generic operations]} 93
\item {обобщенные процедуры [generic procedures]} 167, 171
  \subitem {обобщенный селектор} 177, 179
\item {обработка ошибок [error handling]}
  \subitem {в вычислителе с явным управлением} 512, 516 (упр.~5.30)
  \subitem {в скомпилированном коде} 554{\it п}
\item {обработка потоков [stream processing]} 35{\it п}
\item {обработка сигналов [signal processing]}
  \subitem {переход сигнала через ноль} 323 (упр.~3.74), 323 (упр.~3.75), 324 (упр.~3.76)
  \subitem {потоковая модель} 321
  \subitem {сглаживание сигнала} 323 (упр.~3.75), 324 (упр.~3.76)
  \subitem {сглаживание функции} 88 (упр.~1.44)
\item {образец [pattern]} 409
\item {обратная кавычка [backquote]} 524{\it п}
\item {объект(ы) [object(s)]} 211
  \subitem {преимущества введения в моделирование} 218
  \subitem {с состоянием, меняющимся во времени} 213
\item {объектная программа [object program]} 517
\item {объектно-ориентированные языки программирования [object-oriented pro\-gram\-ming languages]} 195{\it п}
\item {обычные числа [ordinary numbers]}
  \subitem {в обобщенной арифметической системе} 187
\item {ограничения [constraints]}
  \subitem {распространение} 271
  \subitem {элементарные} 272
\item {окружение [environment]} 28, 227
  \subitem {в интерпретаторе запросов} 449 (упр.~4.79)
  \subitem {времени компиляции} {\it см.} окружение времени компиляции
  \subitem {глобальное [global]} 227, {\it см.} глобальное окружение
  \subitem {и лексическая сфера действия} 47
  \subitem {и переименование} 448 (упр.~4.79)
  \subitem {как контекст для вычисления} 30
  \subitem {объемлющее [enclosing]} 227
\item {окружение времени компиляции [compile-time environment]} 549, 549 (упр.~5.40), 549 (упр.~5.41)
  \subitem {и явное кодирование} 550 (упр.~5.44)
\item {операнды комбинации [operands of a combination]} 26
\item {оператор комбинации} 26
  \subitem {в виде \texttt{lambda}-выражения} 75
  \subitem {в виде составного выражения} 39 (упр.~1.4)
  \subitem {как комбинация} 84{\it п}
\item {оператор присваивания [assignment operator]} 212, {\it см.~также} {\tt set!}
\item {операция [operation]}
  \subitem {в регистровой машине} 451
  \subitem {обобщенные} 93
  \subitem {со смешанными типами} 191
\item {определение процедуры [procedure definition]} 31
\item {определения} {\it см.} \texttt{define}; внутренние определения
\item {определенный интеграл [definite integral]} 72
  \subitem {приближенное вычисление методом Монте-Карло} 221 (упр.~3.5), 331 (упр.~3.82)
\item {оптимальность [optimality]}
  \subitem {кода Хаффмана} 165
  \subitem {схемы Горнера} 125{\it п}
\item {освобождение мьютекса} 294
\item {особая форма [special form]} 31
  \subitem {vs. процедура} 371 (упр.~4.26), 379
  \subitem {как производное выражение в интерпретаторе} 346
  \subitem {необходимость} 42 (упр.~1.6)
\item {особые формы}
  \subitem {\texttt{and}} 37
  \subitem {\texttt{begin}} 214
  \subitem {\texttt{cond}} 36
  \subitem {\texttt{cons-stream} {\em (нс)}} 302
  \subitem {\texttt{define}} 28, 32
  \subitem {\texttt{delay} {\em (нс)}} 302
  \subitem {\texttt{if}} 37
  \subitem {\texttt{lambda}} 75
  \subitem {\texttt{let}} 76
  \subitem {\texttt{let*}} 349 (упр.~4.7)
  \subitem {\texttt{letrec}} 363 (упр.~4.20)
  \subitem {\texttt{or}} 37
  \subitem {\texttt{quote}} 147{\it п}
  \subitem {\texttt{set!}} 214
  \subitem {именованный \texttt{let}} 349 (упр.~4.8)
\item {остановки проблема [halting problem]} 360 (упр.~4.15)
\item {остаток по модулю [remainder modulo $n$]} 65
\item {Островский, А.~М.} 125{\it п}
\item {отладка [debugging]} 23
\item {отложенная операция [deferred operation]} 50
\item {относительности теория [theory of relativity]} 298
\item {отношения, вычисления в терминах отношений} 272, 405
\item {отображение [map]} 121
  \subitem {вложенное} 128, 318
  \subitem {деревьев} 119
  \subitem {как преобразователь} 121
  \subitem {списков} 112
\item {отображение координат рамки [frame coordinate map]} 140
\item {отрезок [line segment]}
  \subitem {представление в виде пары векторов} 142 (упр.~2.48)
  \subitem {представление в виде пары точек} 99 (упр.~2.2)
\item {отслеживаемая процедура [monitored procedure]} 217 (упр.~3.2)
\item {очередь [queue]} 250
  \subitem {в плане действий} 269
  \subitem {голова [front]} 250
  \subitem {двусторонняя} 254 (упр.~3.23)
  \subitem {операции} 250
  \subitem {реализация в виде процедуры} 254 (упр.~3.22)
  \subitem {хвост [rear]} 250
\item {ошибка [bug]}
  \subitem {захват свободной переменной} 46
  \subitem {побочного эффекта} 225{\it п}
  \subitem {порядок присваиваний} 226
\item {ошибка округления [roundoff error]} 25{\it п}
\item {ошибки округления} 172{\it п}
\bigskip
\item {пакет [package]} 180
  \subitem {Scheme-number} 187
  \subitem {декартово представление} 181
  \subitem {комплексные числа} 189
  \subitem {многочлены} 200
  \subitem {полярное представление} 181
  \subitem {рациональные числа} 188
\item {память [memory]}
  \subitem {в 1964 г.} 384{\it п}
  \subitem {со списковой структурой [list-structured]} 489
\item {Пан, В.~Я.} 126{\it п}
\item {пара (пары) [pair(s)]} 95
  \subitem {аксиоматическое определение} 100
  \subitem {бесконечные потоки} 318
  \subitem {изменяемые} 242
  \subitem {использование для представления дерева} 115
  \subitem {использование для представления последовательности} 107
  \subitem {ленивые} 379
  \subitem {представленные с помощью векторов} 490
  \subitem {процедурное представление} 100, 249, 379
  \subitem {стрелочная нотация} 106
\item {параллелизм} 281, 282
  \subitem {и функциональное программирование} 333
  \subitem {механизмы для управления} 287
  \subitem {правильность параллельных программ} 285
  \subitem {тупик} 296
\item {параметр [parameter]} {\it см.} формальные параметры
\item {Паскаль, Блез [Blaise Pascal]} 57{\it п}
\item {передача сообщений [message passing]} 101, 184, 185, 216
  \subitem {в банковском счете} 216
  \subitem {в имитаторе цифровых схем} 265
  \subitem {и модель с окружениями} 240 (упр.~3.11)
  \subitem {и хвостовая рекурсия} 51{\it п}
\item {переменная [variable]} 28, {\it см. также} локальная переменная
  \subitem {значение [value]} 28, 227
  \subitem {несвязанная [unbound]} 227
  \subitem {область действия} 45, {\it см. также} область действия переменной
  \subitem {свободная} 45
  \subitem {связанная} 45
\item {переменная образца [pattern variable]} 409
  \subitem {представление} 430, 444
\item {переменная состояния [state variable]} 51, 212
  \subitem {внутренняя} 213
\item {перенаправляющий адрес [forwarding address]} 496
\item {перенос языка (на новый компьютер) [porting a language]} 554
\item {перестановки множества [permutations of a set]} 129
\item {переход сигнала через ноль} 323 (упр.~3.74), 323 (упр.~3.75), 324 (упр.~3.76)
\item {перечислитель [enumerator]} 121
\item {Перлис, Алан~Дж. [Alan~J. Perlis]} 13, 107{\it п}
  \subitem {афоризмы} 27{\it п}, 31{\it п}
\item {печать входных выражений [typing input expressions]} 27{\it п}
\item {печать, элементарные процедуры [primitives for printing]} 96{\it п}
\item {пи ($\pi$)}
  \subitem {аппроксимация через интегрирование методом Монте-Карло} 221 (упр.~3.5)
  \subitem {оценка Чезаро} 219, 330
  \subitem {поток приближений} 315
  \subitem {приближенное вычисление методом половинного деления} 80, 331 (упр.~3.82)
  \subitem {приближенное вычисление через интегрирование методом Монте-Карло} 221 (упр.~3.5), 331 (упр.~3.82)
  \subitem {ряд Лейбница} 70{\it п}, 315
  \subitem {формула Уоллиса} 73 (упр.~1.31)
\item {Пингала, Ачарья [\'Ach\'arya Pingala]} 61{\it п}
\item {Питман, Кент [Kent Pitman]} 24{\it п}
\item {Пифагоровы тройки [Pythagorean triples]}
  \subitem {в недетерминистских программах} 386 (упр.~4.35), 386 (упр.~4.36), 386 (упр.~4.37)
  \subitem {в потоках} 320 (упр.~3.69)
\item {план действий [agenda]} 267, {\it см. также} имитация цифровых схем
\item {плотный многочлен [dense polynomial]} 203
\item {поддержание истины [truth maintenance]} 385{\it п}
\item {подкоренное число [radicand]} 41
\item {подпрограмма,~в регистровой машине [subroutine]} 461
\item {подсказка [prompt]} 356
  \subitem {вычислитель с явным управлением} 512
  \subitem {интерпретатор запросов} 430
  \subitem {ленивый интерпретатор} 374
  \subitem {метациклический интерпретатор} 356
  \subitem {недетерминистский вычислитель} 401
\item {подсказка вывода [output prompt]} 356
\item {подстановка [substitution]} 34{\it п}
\item {подстановочная модель применения процедуры} 33, 34, 227
  \subitem {неадекватность} 222
  \subitem {форма процесса} 49
\item {подсчет команд [instruction counting]} 488 (упр.~5.15)
\item {подтип [subtype]} 194
  \subitem {несколько подтипов} 195
\item {\texttt{подчиняется} (правило)} {\it 414}, 428 (упр.~4.64)
\item {поиск [search]}
  \subitem {в глубину [depth-first]} 384
  \subitem {по бинарным деревьям} 158
  \subitem {систематический [systematic]} 384
\item {поиск с возвратом [backtracking]}
  \subitem {управляемый зависимостями [dependency-directed]} 385{\it п}
  \subitem {хронологический [chronological]} 384
\item {показатель конца списка [end-of-list marker]} 107
\item {поле типа [type field]} 491{\it п}
\item {полноправные элементы вычисления [first-class elements of computation]} 87, {\it см. также} элементы вычисления первого класса
\item {половинного деления метод [half-interval method]} 78
  \subitem {\texttt{half-interval-method}} 80
  \subitem {vs. метод Ньютона} 86{\it п}
\item {полусумматор [half-adder]} 261
  \subitem {\texttt{half-adder}} 262
  \subitem {имитация} 268
\item {пометка с очисткой [mark-sweep]} 495{\it п}
\item {помеченные данные [tagged data]} 175, 490{\it п}
\item {порождение предложений [generating sentences]} 393 (упр.~4.49)
\item {порядок вычисления [order of evaluation]}
  \subitem {в Scheme} 227 (упр.~3.8)
  \subitem {в вычислителе с явным управлением} 505
  \subitem {в компиляторе} 546 (упр.~5.36)
  \subitem {в метациклическом интерпретаторе} 343 (упр.~4.1)
  \subitem {зависимость от реализации} 229{\it п}
  \subitem {и присваивания} 227 (упр.~3.8)
\item {порядок вычисления подвыражений [order of subexpression evaluation]} {\it см.} порядок вычисления
\item {порядок роста [order of growth]} 58
  \subitem {древовидно-рекурсивный процесс} 58
  \subitem {линейно итеративный процесс} 58
  \subitem {линейно рекурсивный процесс} 58
  \subitem {логарифмический} 60
  \subitem {способ записи} 58
\item {порядок событий [order of events]}
  \subitem {неопределенность в параллельных системах} 284
  \subitem {отделение от внешней структуры процедур} 304
\item {последовательное возведение~в квадрат [successive squaring]} 60
\item {последовательности [sequences]} 73{\it п}, 107
  \subitem {как источник модульности} 124
  \subitem {как стандартные интерфейсы} 120
  \subitem {операции} 122
  \subitem {представленные в виде пар} 107
\item {последовательность выражений [sequence of expressions]}
  \subitem {в следствии \texttt{cond}} 37{\it п}
  \subitem {в теле процедуры} 32{\it п}
\item {последовательность команд [instruction sequence]} 521, 536
\item {поток(и) [stream(s)]} 211, 298, 299
  \subitem {бесконечные} {\it см.} бесконечные потоки
  \subitem {и задержанное вычисление} 324
  \subitem {используемые в интерпретаторе языка запросов} 418, 424{\it п}
  \subitem {неявное определение} 309
  \subitem {пустые} 301
  \subitem {реализованные в виде задержанных списков} 299
  \subitem {реализованные в виде ленивых списков} 379
\item {пошаговое написание [incremental development]} 28
\item {пошаговое улучшение [iterative improvement]} 89 (упр.~1.46)
\item {правило (в языке запросов) [rule]} 407, 412
  \subitem {без тела} 413{\it п}, 415
  \subitem {применение} 422, 434, 436, 449 (упр.~4.79)
\item {пре\-ди\-кат}
  \subitem {\texttt{if}} 37
\item {предикат [predicate]} 36
  \subitem {ветви \texttt{cond}} 36
  \subitem {соглашение об именах} 41{\it п}
\item {предложения [statements]} 523, {\it см.} последовательность команд
\item {представление множеств в виде неупорядоченных списков [unordered-list repre\-sen\-ta\-tion of sets]} 155
\item {представление множеств в виде упорядоченных списков [ordered-list repre\-sen\-ta\-tion of sets]} 156
\item {префикс [prefix]} 164
\item {префиксная нотация [prefix notation]} 26
  \subitem {vs. инфиксная нотация} 154 (упр.~2.58)
\item {префиксный код [prefix code]} 164
\item {приведение к наименьшему знаменателю [reducing to lowest terms]} 97, 98, 209
\item {приведение типов [coercion]} 192
  \subitem {в алгебраических манипуляциях} 206
  \subitem {в арифметике многочленов} 202
  \subitem {процедура} 192
  \subitem {таблица} 192
\item {применение процедур [procedure application]}
  \subitem {модель вычисления с окружениями} 231
  \subitem {обозначение комбинаций} 26
  \subitem {подстановочная модель} {\it см.} подстановочная модель применения процедуры
\item {присваивание [assignment]} 212, {\it см.~также} {\tt set!}
  \subitem {вызванные ошибки} 225{\it п}, 226
  \subitem {преимущества} 218
  \subitem {расплата за} 221
\item {провод, в цифровой схеме [wire]} 261
\item {программа [program]} 22
  \subitem {как абстрактная машина} 357
  \subitem {как данные} 357
  \subitem {комментарии} 129{\it п}
  \subitem {пошаговое написание} 28
  \subitem {структура} 28, 44, 46, {\it см. также} барьеры абстракции
  \subitem {структурирование с помощью подпрограмм} 359{\it п}
\item {программирование [programming]}
  \subitem {императивное} 225
  \subitem {управляемое данными} {\it см.} программирование, управляемое данными
  \subitem {управляемое потребностями} 304
  \subitem {функциональное} {\it см.} функциональное программирование
  \subitem {чрезвычайно плохой стиль} 306{\it п}
  \subitem {элементы} 25
\item {программирование, управляемое данными [data-directed programming]} 93, 171, 179, 180
  \subitem {vs. разбор случаев} 341
  \subitem {в интерпретаторе запросов} 431
  \subitem {в метациклическом интерпретаторе} 348 (упр.~4.3)
\item {продолжение [continuation]}
  \subitem {в имитаторе регистровых машин} 478{\it п}
  \subitem {в недетерминистском интерпретаторе} 394, 396, {\it см. также} продолжение неудачи, продолжение успеха
\item {продолжение неудачи [failure continuation]} 394, 396
  \subitem {{\tt amb}} 401
  \subitem {присваивания} 399
  \subitem {управляющий цикл} 401
\item {продолжение успеха [success continuation]} 394, 396
\item {прозрачность референциальная} 224
\item {производная функции} 85
\item {производные выражения в интерпретаторе [derived expressions]} 348
\item {производные выражения~в интерпретаторе}
  \subitem {в вычислителе с явным управлением} 511 (упр.~5.23)
\item {простой запрос} 409
  \subitem {обработка} 418, 423, 431
\item {простые числа} 64
  \subitem {бесконечный поток} {\it см.} \texttt{primes}
  \subitem {и криптография} 67{\it п}
  \subitem {проверка на простоту} 64
  \subitem {решето Эратосфена} 308
  \subitem {тест Миллера-Рабина} 69 (упр.~1.28)
  \subitem {тест Ферма} 65
\item {процедура [procedure]} 24, 25
  \subitem {vs. математическая функция} 40
  \subitem {vs. особая форма} 371 (упр.~4.26), 379
  \subitem {безымянная} 75
  \subitem {в качестве аргумента} 70
  \subitem {возврат нескольких значений} 478{\it п}
  \subitem {высшего порядка} {\it см.} процедура высшего порядка
  \subitem {именование (с помощью \texttt{define})} 32
  \subitem {имя} 32
  \subitem {как возвращаемое значение} 83
  \subitem {как данные} 24
  \subitem {как обобщенный метод} 78
  \subitem {как черный ящик} 44
  \subitem {как шаблон локальной эволюции процесса} 48
  \subitem {мемоизированная} 260 (упр.~3.27)
  \subitem {нестрогая [non-strict]} 371
  \subitem {неявный \texttt{begin} в теле} 214{\it п}
  \subitem {область действия формальных параметров} 45
  \subitem {обобщенная} 167, 171
  \subitem {определение} 31
  \subitem {отслеживаемая} 217 (упр.~3.2)
  \subitem {полноправный статус~в Лиспе} 87
  \subitem {произвольное количество аргументов} 26, 112 (упр.~2.20)
  \subitem {создание с помощью \texttt{define}} 32
  \subitem {создание с помощью {\tt lambda}} 229, 230
  \subitem {создание~с помощью {\tt lambda}} 75
  \subitem {составная [compound]} 31
  \subitem {тело} 32
  \subitem {формальные параметры} 32
\item {процедура высшего порядка [higher-order procedure]} 70
  \subitem {в метациклическом интерпретаторе} 341{\it п}
  \subitem {и сильная типизация} 328{\it п}
  \subitem {процедура как возвращаемое значение} 83
  \subitem {процедура как обобщенный метод} 78
  \subitem {процедура~в качестве аргумента} 70
\item {процедурная абстракция [procedural abstraction]} 44
\item {процедурное представление данных [procedural representation of data]} 100
  \subitem {изменяемые данные} 249
\item {процесс [process]} 22
  \subitem {древовидно-рекурсивный} 53
  \subitem {итеративный} 50
  \subitem {линейно итеративный} 51
  \subitem {линейно рекурсивный} 50
  \subitem {локальная эволюция} 48
  \subitem {необходимые ресурсы} 58
  \subitem {порядок роста} 58
  \subitem {рекурсивный} 50
  \subitem {форма} 50
\item {прочесывание внутренних определений [scanning out internal definitions]} 361
  \subitem {в компиляторе} 549{\it п}, 550 (упр.~5.43)
\item {прямоугольники, их представление [representing rectangles]} 99 (упр.~2.3)
\item {псевдоделение многочленов [pseudodivision]} 209
\item {псевдоним [aliasing]} 225{\it п}
\item {псевдоостаток многочленов [pseudoremainder]} 209
\item {псевдослучайная последовательность чисел [pseudo-random sequence]} 218{\it п}
\item {пустой поток [empty stream]} 301
\item {пустой список [empty list]} 109
  \subitem {обозначенный {\tt '()}} 109{\it п}
  \subitem {распознавание с помощью {\tt null?}} 110
\item {пути данных регистровой машины [data paths for register machine]} 451
  \subitem {диаграмма путей данных} 452
\bigskip
\item {Рабин, Майкл~О. [Michael~O. Rabin]} 69 (упр.~1.28)
\item {равенство [equality]}
  \subitem {в обобщенной арифметической системе} 191 (упр.~2.79)
  \subitem {и референциальная прозрачность} 224
  \subitem {символов} 148{\it п}
  \subitem {списков} 149 (упр.~2.54)
  \subitem {чисел} 36, 149{\it п}, 491{\it п}
\item {равенство по модулю $n$ [congruency modulo $n$]} 65
\item {радиус интервала [width, of an interval]} 104 (упр.~2.9)
\item {разбитое сердце [broken heart]} 496
\item {разбор случаев [case analysis]} 36
  \subitem {vs. стиль, управляемый данными} 341
  \subitem {обший} {\it см. также} \texttt{cond}
  \subitem {с двумя случаями} 37
\item {разделение времени [time slicing]} 296
\item {разделяемое состояние [shared state]} 285
\item {разделяемые данные} 246
\item {разделяемые ресурсы [shared resources]} 291
\item {размен денег [counting change]} 55, 111 (упр.~2.19)
\item {разреженный многочлен [sparse polynomial]} 203
\item {Райт, Джесси~Б. [Jesse~B. Wright]} 100{\it п}
\item {Райт,~Э.~М. [E.~M.~Wright]} 311{\it п}
\item {рак точки с запятой [cancer of the semicolon]} 31{\it п}
\item {Рамануджан, Шриниваса [Srinivasa Ramanujan]} 321{\it п}
\item {Рамануджана числа [Ramanujan numbers]} 321 (упр.~3.71)
\item {рамка [frame]} 133, 140
  \subitem {отображение координат} 140
\item {Ранкл, Джон Дэниел [John Daniel Runkle]} 133{\it п}
\item {распространение ограничений [propagation of constraints]} 271
\item {Рафаэль, Бертрам [Bertram Raphael]} 405{\it п}
\item {рациональная арифметика [rational-number arithmetic]} 93
  \subitem {взаимодействие с обобщенной арифметической системой} 188
  \subitem {необходимость составных данных} 91
\item {рациональная функция [rational function]} 207
  \subitem {приведение к наименьшему знаменателю} 209
\item {рациональные числа [rational numbers]}
  \subitem {арифметические операции} 93
  \subitem {в MIT Scheme} 42{\it п}
  \subitem {печать} 96
  \subitem {представление в виде пар} 96
  \subitem {приведение к наименьшему знаменателю} 97, 98
\item {регистр [register]} 450
  \subitem {представление} 474
\item {регистровая машина [register machine]} 450
  \subitem {действия} 456
  \subitem {диаграмма контроллера} 452
  \subitem {диаграмма путей данных} 452
  \subitem {контроллер} 451
  \subitem {отслеживание производительности} 486
  \subitem {подпрограмма} 459
  \subitem {программа моделирования} 472
  \subitem {проектирование} 451
  \subitem {пути данных} 451
  \subitem {стек} 463
  \subitem {тест} 452
  \subitem {трассировка} 488 (упр.~5.18)
  \subitem {язык для описания} 454
\item {резолюции принцип [resolution principle]} 405{\it п}
  \subitem {Хорновские формы} 405{\it п}
\item {Рейтер, Андреас [Andreas Reuter]} 297{\it п}
\item {рекурсивная процедура [recursive procedure]}
  \subitem {vs. рекурсивный процесс} 51
  \subitem {определение рекурсивной процедуры} 43
  \subitem {построение без \texttt{define}} 364 (упр.~4.21)
\item {рекурсивный процесс [recursive process]} 50
  \subitem {vs. итеративный процесс} 48, 233 (упр.~3.9), 463, 544 (упр.~5.34)
  \subitem {vs. рекурсивная процедура} 51
  \subitem {древовидный} 53, 58
  \subitem {линейный} 50, 58
  \subitem {регистровая машина} 463
\item {рекурсии теория [recursion theory]} 359{\it п}
\item {рекурсия [recursion]} 29, 43
  \subitem {в правилах} 414
  \subitem {выражение сложного процесса} 29
  \subitem {для работы с деревьями} 115
  \subitem {управляемая данными} 203
\item {референциальная прозрачность} 224
\item {решение уравнений [solving equations]} {\it см.} Ньютона метод; половинного деления метод; \texttt{solve}
\item {решето Эратосфена [sieve of Eratosthenes]} 308
  \subitem {\texttt {sieve}} 308
\item {Ривест, Рональд~Л. [Ronald~L. Rivest]} 67{\it п}, 160{\it п}
\item {Рис, Джонатан~А. [Jonathan~A. Rees]} 348{\it п}, 366{\it п}
\item {рисовалка [painter]} 133
  \subitem {операции} 133
  \subitem {операции высших порядков} 137
  \subitem {представление в виде процедуры} 141
  \subitem {преобразование и комбинирование} 143
\item {Робинсон, Дж.~А. [J.~A.~Robinson]} 405{\it п}
\item {Роджерс, Уильям Бартон [William Barton Rogers]} 133{\it п}
\item {рок-песни 1950-х гг [1950s rock songs]} 169 (упр.~2.70)
\item {Росас, Гильермо Хуан [Guillermo Juan Rozas]} 535{\it п}
\item {русского крестьянина метод умножения [Russian peasant method of mul\-ti\-pli\-ca\-tion]} 61{\it п}
\item {Рэймонд, Эрик [Eric Raymond]} 370{\it п}, 384{\it п}
\bigskip
\item {с\texttt{cons}ить [{\tt cons} up]} 110
\item {Савин,~ А.Н.} 22
\item {Сазерленд, Айвен [Ivan Sutherland]} 272{\it п}
\item {самовычисляющееся выражение [self-evaluating expression]} 340
\item {самосознание, повышение уровня [expansion of consciousness]} 342{\it п}
\item {санк [thunk]} 372
  \subitem {вызова по имени [call-by-name]} 306{\it п}
  \subitem {вызова по необходимости [call-by-need]} 306{\it п}
  \subitem {вынуждение} 372
  \subitem {происхождение названия} 372{\it п}
  \subitem {реализация} 375
\item {Сассман, Джеральд Джей [Gerald Jay Sussman]} 24{\it п}, 272{\it п}, 385{\it п}
\item {Сассман, Джули Эстер Мазель, племянницы [nieces of Julie Esther Mazel Sussman]} 146
\item {сбалансированное бинарное дерево [balanced binary tree]} 160, {\it см.~также} бинарное дерево
\item {сбалансированный мобиль [balanced mobile]} 118 (упр.~2.29)
\item {сборка мусора [garbage collection]} 375{\it п}, 489, 494, 495
  \subitem {и изменения данных} 242{\it п}
  \subitem {и мемоизация} 375{\it п}
  \subitem {и хвостовая рекурсия} 535{\it п}
  \subitem {сжимающая [compacting]} 496{\it п}
  \subitem {через остановку с копированием [stop-and-copy]} 495
  \subitem {через пометку с очисткой [mark-sweep]} 495{\it п}
\item {сборщик мусора [garbage collector]} 242{\it п}
  \subitem {остановка с копированием} 495
  \subitem {сжимающий} 496{\it п}
  \subitem {через пометку с очисткой} 495{\it п}
\item {свободная переменная [free variable]} 45
  \subitem {захват} 46
  \subitem {лексическая сфера действия} 47
\item {связанная переменная [bound variable]} 45
\item {связывание [binding]} 45, 227
  \subitem {глубокое [deep]} 354{\it п}
\item {сглаживание сигнала [smoothing a signal]} 323 (упр.~3.75), 324 (упр.~3.76)
\item {сглаживание функции} 88 (упр.~1.44)
\item {секретарь, его важное значение [importance of a secretary]} 408
\item {селектор [selector]} 93
  \subitem {как барьер абстракции} 98
  \subitem {обобщенный} 177, 179
\item {семафор [semaphore]} 294{\it п}
  \subitem {размера $n$} 296 (упр.~3.47)
\item {сериализатор [serializer]} 288, 289
  \subitem {реализация} 294
  \subitem {с множественными разделяемыми ресурсами} 291
\item {сеть ограничений [constraint network]} 272
\item {сжимающий сборщик мусора [compacting garbage collector]} 496{\it п}
\item {сигма-запись ($\sum$)} 71
\item {сильно типизированный язык [strongly typed language]} 328{\it п}
\item {символ в коде ASCII [character in ASCII encoding]} 163
\item {символ(ы) [symbol(s)]} 146
  \subitem {восприятие} 492
  \subitem {кавычки} 147
  \subitem {представление} 491
  \subitem {равенство} 148
  \subitem {уникальность} 247{\it п}
\item {символьная алгебра [symbolic algebra]} 198
\item {символьное выражение [symbolic expression]} 92, {\it см. также} символ(ы)
\item {символьное дифференцирование [symbolic differentiation]} 149, 183 (упр.~2.73)
\item {Симпсона правило для численного интегрирования [Simpson's Rule for nu\-me\-ri\-cal integration]} 73 (упр.~1.29)
\item {синтаксис [syntax]} 339, {\it см. также} особые формы
  \subitem {абстрактный} {\it см.} абстрактный синтаксис
  \subitem {выражений, описание} 32{\it п}
  \subitem {языка программирования} 31
\item {синтаксический анализ естественного языка [parsing]} 389
  \subitem {настоящая обработка языка vs. игрушечный анализатор} 393{\it п}
\item {синтаксический анализ, отделение от выполнения [separating syntactic analysis from execution]}
  \subitem {в имитаторе регистровых машин} 476, 481
  \subitem {в метациклическом интерпретаторе} 365
\item {синтаксический интерфейс [syntax interface]} 266{\it п}
\item {синтаксический сахар [syntactic sugar]} 31{\it п}
  \subitem {\texttt{define}} 344
  \subitem {\texttt{let}} 77
  \subitem {процедура vs. данные} 266{\it п}
  \subitem {циклические конструкции} 51
\item {синус [sine]}
  \subitem {приближение при малых углах} 59 (упр.~1.15)
  \subitem {степенной ряд} 312 (упр.~3.59)
\item {синхронизация} {\it см.} параллелизм
\item {систематический поиск [systematic search]} 384
\item {скобки [parentheses]}
  \subitem {в определении процедуры} 32
  \subitem {обозначающие клаузу \texttt{cond}} 36
  \subitem {обозначение применения функции~к аргументам} 26
\item {слизывание [snarfing]} 370{\it п}
\item {слияние [merge]} 334
\item {слияние бесконечных потоков [merging infinite streams]} {\it см.} бесконечные потоки
\item {смещение [displacement]} 548
\item {собака идеально разумная, поведение [behavior of a perfectly rational dog]} 296{\it п}
\item {совместимое расширение [upward-compatible extension]} 378 (упр.~4.31)
\item {согласованность кэша [cache coherence]} 285{\it п}
\item {соглашение об именах [naming conventions]}
  \subitem {\texttt{!} для присваиваний и изменений} 214{\it п}
  \subitem {\texttt{?} для предикатов} 41{\it п}
\item {соединитель (соединители) в системе ограничений [connector(s), in constraint system]} 272
  \subitem {операции} 275
  \subitem {представление} 278
\item {сокрытие связывания [shadowing a binding]} 228
\item {сокрытия принцип [hiding principle]} 215{\it п}
\item {Соломонофф, Рэй [Ray Solomonoff]} 218{\it п}
\item {сопоставитель [pattern matcher]} 417
\item {сопоставление с образцом [pattern matching]} 417
  \subitem {vs. унификация} 422, 424{\it п}
  \subitem {реализация} 434
\item {сопротивление [resistance]}
  \subitem {погрешность} 102
  \subitem {формула для параллельных резисторов} 102, 105
\item {составная процедура [compound procedure]} 31, {\it см. также} процедура
  \subitem {использование~в качестве элементарных} 33
\item {составное выражение [compound expression]} 26, {\it см. также} комбинация; особая форма
  \subitem {как оператор комбинации} 39 (упр.~1.4)
\item {составной запрос [compound query]} 411
  \subitem {обработка} 419, 432, 447 (упр.~4.75), 448 (упр.~4.76), 448 (упр.~4.77)
\item {составной объект данных [compound data object]} 91
\item {составные данные, их необходимость [compound data, need for]} 90
\item {состояние [state]}
  \subitem {внутреннее} {\it см.} внутреннее состояние
  \subitem {исчезновение в потоковой формулировке} 332
  \subitem {разделяемое} 285
\item {Спаффорд, Юджин~Г. [Eugene~H. Spafford]} 554{\it п}
\item {специалист по численному анализу [numerical analyst]} 79{\it п}
\item {списковая структура} 96, 108{\it п}
  \subitem {vs. список} 108{\it п}
  \subitem {изменяемая} 242
  \subitem {представленная с помощью векторов} 490
\item {список [list]} 108, 108{\it п}
  \subitem {$n$-й элемент} 109
  \subitem {vs. списковая структура} 108{\it п}
  \subitem {длина} 110
  \subitem {кавычки} 147
  \subitem {ленивый} 379
  \subitem {манипуляции с {\tt car}, {\tt cdr} и {\tt cons}} 108
  \subitem {методы работы} 109
  \subitem {обратная кавычка} 524{\it п}
  \subitem {обращение} 111 (упр.~2.18)
  \subitem {операции} 109
  \subitem {отображение} 112
  \subitem {последняя пара} 111 (упр.~2.17)
  \subitem {представление на печати} 108
  \subitem {преобразование в бинарное дерево} 161 (упр.~2.64)
  \subitem {преобразование из бинарного дерева} 161 (упр.~2.63)
  \subitem {пустой} {\it см.} пустой список
  \subitem {равенство} 149 (упр.~2.54)
  \subitem {с заголовком [headed]} 255, 269{\it п}
  \subitem {<<с\texttt{cons}ивание>>} 110
  \subitem {соединение через \texttt{append}} 110
  \subitem {<<у\texttt{cdr}ивание>>} 109
\item {список свободных ячеек [free list]} 493{\it п}
\item {список термов многочлена [term list of polynomial]} 199, 200
  \subitem {представление} 203
\item {средства абстракции [means of abstraction]} 25
  \subitem {\texttt{define}} 28
\item {средства комбинирования [means of cvombination]} 25, {\it см. также} замыкание
\item {стандартный интерфейс [conventional interface]} 92, 120
  \subitem {последовательность} 120
\item {стек [stack]} 51{\it п}, 467
  \subitem {для рекурсии~в регистровой машине} 463
  \subitem {кадрированный} 503{\it п}
  \subitem {представление} 474, 493
\item {степенной ряд, представленный как поток [power series, as a stream]} 312 (упр.~3.59)
  \subitem {деление} 314 (упр.~3.62)
  \subitem {перемножение} 313 (упр.~3.60)
  \subitem {сложение} 313 (упр.~3.60)
\item {степенные ряды как потоки}
  \subitem {интегрирование} 313 (упр.~3.59)
\item {Стил, Гай Льюис мл. [Guy Lewis Steele Jr.]} 24{\it п}, 52{\it п}, 226{\it п}, 272{\it п}, 370{\it п}, 384{\it п}
\item {стиль, ориентированный на выражения, vs. императивный стиль [expression-oriented style vs. imperative style]} 281{\it п}
\item {Стой, Джозеф~Э. [Joseph~E. Stoy]} 34{\it п}, 62{\it п}, 364{\it п}
\item {Столлман, Ричард~М. [Richard~M. Stallman]} 272{\it п}, 385{\it п}
\item {Стрейчи, Кристофер [Christopher Strachey]} 87{\it п}
\item {стрелка [pointer]} 106
  \subitem {в стрелочной диаграмме} 106
\item {стрелочная диаграмма [box-and-pointer notation]} 106
\item {строгая процедура [procedure]} 371
\item {строка [string]}
  \subitem {печатание} 147{\it п}
  \subitem {элементарные процедуры} 444{\it п}, 527{\it п}
\item {структура управления [control structure]} 425
\item {сумматор [full-adder]} 263
  \subitem {\texttt{full-adder}} 263
  \subitem {каскадный [ripple-carry]} 265 (упр.~3.30)
  \subitem {полу-} 261
\item {сумматор для сигналов [summer]} 322
\item {суммирование последовательности [summation of a series]} 71
  \subitem {с потоками} 315
  \subitem {ускорение последовательности приближений} 315
\item {схема [circuit]}
  \subitem {моделируемая с помощью потоков} 322 (упр.~3.73), 328 (упр.~3.80)
  \subitem {цифровая} {\it см.} имитация цифровых схем
\item {схема Горнера [Horner's rule]} 125 (упр.~2.34)
\item {счет банковский} {\it см.} банковский счет
\item {счетчик программы [program counter]} 475
\bigskip
\item {таблица [table]} 254
  \subitem {$n$-мерная} 259 (упр.~3.25)
  \subitem {двумерная} 256
  \subitem {для программирования, управляемого данными} 180
  \subitem {используемая для моделирования плана действий} 269
  \subitem {используемая для хранения вычисленных значений} 260 (упр.~3.27)
  \subitem {локальная} 258
  \subitem {одномерная} 255
  \subitem {операций и типов} {\it см.} таблица операций и типов
  \subitem {представленная в виде бинарного дерева vs. неупорядоченного списка} 260 (упр.~3.26)
  \subitem {приведения} 192
  \subitem {проверка равенства ключей} 259 (упр.~3.24)
  \subitem {хребет [backbone]} 255
\item {таблица операций и типов [operation-and-type table]} 180
  \subitem {необходимость присваивания} 213{\it п}
  \subitem {реализация} 259
\item {таблица регистров,~в программе моделирования [register table]} 475
\item {табло [tableau]} 316
\item {табуляризация [tabulation]} 57{\it п}, 260 (упр.~3.27)
\item {тангенс [tangent]}
  \subitem {как цепная дробь} 83 (упр.~1.39)
  \subitem {степенной ряд} 314 (упр.~3.62)
\item {тег типа [type tag]} 491{\it п}
\item {теговая архитектура [tagged architecture]} 491{\it п}
\item {Тейтельман, Уоррен [Warren Teitelman]} 24{\it п}
\item {текущее время, для имитации плана [current time]} 269
\item {тело процедуры [body of a procedure]} 32
\item {теорема об остановке [Halting Theorem]} 360{\it п}
\item {Т\"ернер, Дэвид [David Turner]} 128{\it п}, 319{\it п}, 333{\it п}
\item {тест на равенство нулю (обобщенный) [zero test]} 191 (упр.~2.80)
  \subitem {для многочленов} 204 (упр.~2.87)
\item {тест, операция~в регистровой машине [test]} 452
\item {тета от $f(n)$ ($\Theta(f(n))$)} 58
\item {техника vs. математика [engineering vs. mathematics]} 66{\it п}
\item {тип данных}
  \subitem {башня [tower of types]} 194
  \subitem {в Лиспе} 191 (упр.~2.78)
  \subitem {в сильно типизированных языках} 328{\it п}
  \subitem {диспетчеризация} 179
  \subitem {иерархия в символьной алгебре} 206
  \subitem {надтип} 194
  \subitem {несколько подтипов и надтипов} 195
  \subitem {операции со смешанными типами} 191
  \subitem {подтип} 194
  \subitem {подъем} 194, 197 (упр.~2.83)
  \subitem {спуск} 195, 198 (упр.~2.85)
\item {тип связи [linkage descriptor]} 520
\item {типизированный указатель [typed pointer]} 490
\item {тождественность и изменение [sameness and change]}
  \subitem {значение} 223
  \subitem {и разделяемые данные} 246
\item {торможение усреднением [average damping]} 82
\item {точечная запись [dotted-tail notation]}
  \subitem {в образцах запросов} 410, 435
  \subitem {в правилах языка запросов} 415
  \subitem {для процедурных параметров} 112 (упр.~2.20), 182{\it п}
  \subitem {и {\tt read}} 436
\item {точка входа [entry point]} 454
\item {точка недетерминистского выбора [nondeterministic choice point]} 384
\item {точка с запятой [semicolon]} 31{\it п}
  \subitem {как символ комментария} 129{\it п}
\item {точка, представленная в виде пары чисел [point, represented as a pair]} 99 (упр.~2.2)
\item {точные целые числа [exact integers]} 42{\it п}
\item {трассировка [tracing]}
  \subitem {команд [instruction]} 488 (упр.~5.16)
  \subitem {регистров [register]} 488 (упр.~5.18)
\item {требуемые регистры [needed registers]} {\it см.} последовательность команд
\item {треугольник Паскаля [Pascal's triangle]} 57 (упр.~1.12)
\item {тригонометрические тождества [trigonometric relations]} 175
\item {тупик} 296, 297
  \subitem {выход} 297{\it п}
  \subitem {способ избежания} 297
\item {Тьюринг, Алан~М. [Alan~M. Turing]} 359{\it п}, 360{\it п}
\item {Тэтчер, Джеймс~У. [James~W. Thatcher]} 100{\it п}
\bigskip
\item {Уайз, Дэвид~С. [David~S. Wise]} 306{\it п}
\item {Уайлс, Эндрю [Andrew Wiles]} 65{\it п}
\item {Уайльд, Оскар, парафраза Перлиса [Oscar Wilde]} 27{\it п}
\item {уголь, битумный [bituminous coal]} 133{\it п}
\item {удовлетворение запросу [satisfying a pattern]} 410
  \subitem {составному} 411
\item {Уиздом, Джек [Jack Wisdom]} 24{\it п}
\item {Уинстон, Патрик Генри [Patrick Henry Winston]} 385{\it п}, 393{\it п}
\item {указатель [pointer]}
  \subitem {типизированный [typed]} 490
\item {умножение методом русского крестьянина [Russian peasant method of mul\-ti\-pli\-ca\-tion]} 61{\it п}
\item {универсальная машина [universal machine]} 359
  \subitem {вычислитель с явным управлением} 517
  \subitem {компьютер общего назначения} 517
\item {унификация [unification]} 405, 417, 421
  \subitem {vs. сопоставление с образцом} 422, 424{\it п}
  \subitem {открытие алгоритма} 405{\it п}
  \subitem {реализация} 437
\item {Уодлер, Филип [Philip Wadler]} 225{\it п}
\item {Уодсворт, Кристофер [Christopher Wadsworth]} 329{\it п}
\item {Уокер, Фрэнсис Амаса [Francis Amasa Walker]} 133{\it п}
\item {Уоллис, Джон [John Wallis]} 73{\it п}
\item {Уотерс, Ричард~К. [Richard~C. Waters]} 125{\it п}
\item {управляющий цикл [driver loop]} 356
  \subitem {в вычислителе с явным управлением} 512
  \subitem {в интерпретаторе запросов} 424, 429
  \subitem {в ленивом интерпретаторе} 374
  \subitem {в метациклическом интерпретаторе} 356
  \subitem {в недетерминистском интерпретаторе} 385, 401
\item {упрощение алгебраических выражений [simplification of algebraic expressions]} 152
\item {уравнения рекурсии [recursion equations]} 23
\item {уравнения, решения} {\it см.} метод половинного деления; Ньютона метод; \texttt{solve}
\item {уровневое проектирование [stratified design]} 145
\item {ускоритель последовательности [sequence accelerator]} 315
\item {условное выражение [conditional expression]}
  \subitem {{\tt cond}} 36
  \subitem {{\tt if}} 37
\item {устойчивость программы [robustness]} 145
\item {утверждение [assertion]} 407
  \subitem {неявное} 413
\item {утренняя звезда [morning star]} {\it см.} Венера
\bigskip
\item {факториал [factorial]} 48, {\it см. также} \texttt{factorial}
  \subitem {без \texttt{letrec} и \texttt{define}} 364 (упр.~4.21)
  \subitem {бесконечный поток} 311 (упр.~3.54)
  \subitem {с \texttt{letrec}} 364 (упр.~4.20)
\item {Фейгенбаум, Эдвард [Edward Feigenbaum]} 406{\it п}
\item {Феничель, Роберт [Robert Fenichel]} 495{\it п}
\item {Ферма Малая теорема [Fermat's Little Theorem]} 65
  \subitem {альтернативная формулировка} 69 (упр.~1.28)
  \subitem {доказательство} 65{\it п}
\item {Ферма тест на простоту [Fermat test for primality]} 65
  \subitem {вариант} 69 (упр.~1.28)
\item {Ферма, Пьер де [Pierre de Fermat]} 65{\it п}
\item {Фибоначчи числа [Fibonacci numbers]} 53, {\it см. также} \texttt{fib}
  \subitem {бесконечный поток} {\it см.} \texttt{fibs}
  \subitem {и алгоритм Евклида для НОД} 63
\item {Фили, Марк [Marc Feeley]} 366{\it п}
\item {Филипс, Хьюберт [Hubert Phillips]} 388 (упр.~4.42)
\item {фильтр [filter]} 74 (упр.~1.33), 121
\item {Флойд, Роберт [Robert Floyd]} 385{\it п}
\item {Форбус, Кеннет~Д. [Kenneth~D. Forbus]} 385{\it п}
\item {форма процесса [shape of a process]} 50
\item {формальные параметры процедуры [formal parameters]} 32
  \subitem {имена} 45
  \subitem {область действия} 45
\item {форматирование входных выражений [formatting input expressions]} 27{\it п}
\item {Фридман, Дэниел~П. [Daniel~P. Friedman]} 306{\it п}, 337{\it п}
\item {функциональное программирование [functional programming]} 222, 330
  \subitem {и время} 331
  \subitem {и параллельность} 333
  \subitem {функциональный язык программирования} 333
\item {функциональный элемент в цифровых схемах [function box]} 261
\item {функциональный язык программирования [functional programming language]} 333
\item {функция (математическая) [function]}
  \subitem {$\mapsto$-нотация} 81{\it п}
  \subitem {vs. процедура} 40
  \subitem {Аккермана} 52 (упр.~1.10)
  \subitem {композиция} 88 (упр.~1.42)
  \subitem {многократное применение} 88 (упр.~1.43)
  \subitem {неподвижная точка} 80
  \subitem {производная} 85
  \subitem {рациональная [rational]} 207
  \subitem {сглаживание} 88 (упр.~1.44)
\bigskip
\item {Хайтин, Грегори [Gregory Chaitin]} 218{\it п}
\item {Хайям, Омар} 57{\it п}
\item {Хансон, Кристофер~П. [Cristopher~P. Hanson]} 348{\it п}, 535{\it п}
\item {хаос~в динамике Солнечной системы [chaos in Solar system]} 24{\it п}
\item {Харди, Годфри Харольд [Godfrey Harold Hardy]} 311{\it п}, 321{\it п}
\item {Хаффман, Дэвид [David Huffman]} 164
\item {Хаффмана код [Huffman code]} 163, {\it см.} код Хаффмана
  \subitem {оптимальность} 165
  \subitem {порядок роста для кодирования} 170 (упр.~2.72)
\item {хвостовая рекурсия [tail recursion]} 51, 508
  \subitem {в Scheme} 52{\it п}
  \subitem {и вычислитель с явным управлением} 508, 515 (упр.~5.26), 516 (упр.~5.28)
  \subitem {и компилятор} 535
  \subitem {и метациклический интерпретатор} 508
  \subitem {и модель с окружениями} 234{\it п}
  \subitem {и сборка мусора} 535{\it п}
\item {хвостовая рекурсия~в списке аргументов [evlis tail recursion]} 505{\it п}
\item {Хейвендер, Дж. [J.~Havender]} 297{\it п}
\item {Хейнс, Кристофер~Т. [Christopher~T. Haynes]} 337{\it п}
\item {Хендерсон, Питер [Peter Henderson]} 132{\it п}, 308{\it п}, 333{\it п}
  \subitem {хендерсоновская диаграмма} 308
\item {Херн, Энтони~К. [Anthony~C. Hearn]} 24{\it п}
\item {Хилфингер, Пол [Paul Hilfinger]} 162{\it п}
\item {Хоар, Чарльз Энтони Ричард [Charles Anthony Richard Hoare]} 100{\it п}
\item {Ходжес, Эндрю [Andrew Hodges]} 359{\it п}
\item {Хофштадтер, Дуглас~Р. [Douglas~R. Hofstadter]} 359{\it п}
\item {Хьюз,~Р.~Дж.~М. [R.~J.~M.~Hughes]} 380{\it п}
\item {Хьюитт, Карл Эдди [Carl Eddie Hewitt]} 51{\it п}, 385{\it п}, 405{\it п}, 495{\it п}
\item {Хэмминг, Ричард Уэсли [Richard Wesley Hamming]} 165{\it п}, 311 (упр.~3.56)
\bigskip
\item {целевой регистр [target]} 520
\item {целые числа [integers]} 25{\it п}
  \subitem {деление} 42{\it п}
  \subitem {точные} 42{\it п}
\item {цепная дробь [continued fraction]} 82 (упр.~1.37)
  \subitem {$e$} 83 (упр.~1.38)
  \subitem {золотое сечение} 82 (упр.~1.37)
  \subitem {тангенс} 83 (упр.~1.39)
\item {цепь электрическая [circuit]}
  \subitem {моделируемая с помощью потоков} 322 (упр.~3.73), 328 (упр.~3.80)
\item {Цзю Ши-Цзе [Chu Shih-Chieh]} 57{\it п}
\item {цикл в списке [cycle in list]} 245 (упр.~3.13)
  \subitem {обнаружение} 248 (упр.~3.18)
\item {цикл обратной связи, моделируемый с помощью потоков [feedback loop]} 324
\item {цикл чтение-вычисление-печать [read-eval-print loop]} 27, {\it см. также} управляющий цикл
\item {циклические конструкции [looping constructs]} 42, 51
  \subitem {реализация в метациклическом интерпретаторе} 350 (упр.~4.9)
\item {цифровой сигнал [digital signal]} 261
\bigskip
\item {Чанда-сутра} 61{\it п}
\item {Чарняк, Юджин [Eugene Charniak]} 385{\it п}
\item {Чебышев, Пафнутий Львович} 311{\it п}
\item {Чезаро, Эрнесто [Ernesto Ces\`aro]} 219{\it п}
\item {Чепман, Дэвид [David Chapman]} 385{\it п}
\item {<<Червь>>~в Интернете [Internet ``worm'']} 554{\it п}
\item {черный ящик [black box]} 44
\item {Ч\"ерч, Алонсо [Alonzo Church]} 75{\it п}, 102 (упр.~2.6)
\item {Ч\"ерча числа [Church numerals]} 102 (упр.~2.6)
\item {Ч\"ерча-Тьюринга тезис [Church-Turing thesis]} 359{\it п}
\item {чисел теория [number theory]} 65{\it п}
\item {числа [numbers]}
  \subitem {в Лиспе} 26
  \subitem {в обобщенной арифметической системе} 187
  \subitem {вещественные [real]} 25{\it п}
  \subitem {десятичная точка} 42{\it п}
  \subitem {зависимость от реализации} 42{\it п}
  \subitem {простые по отношению~к другому числу} 74 (упр.~1.33)
  \subitem {равенство} 36, 149{\it п}, 491{\it п}
  \subitem {рациональные} 42{\it п}
  \subitem {сравнение} 36
  \subitem {точные целые} 42{\it п}
  \subitem {целые [integer]} 25{\it п}
  \subitem {целые vs. вещественные} 25{\it п}
\item {численные данные [numerical data]} 25
\item {численный анализ [numerical analysis]} 25{\it п}
\item {``что такое'' vs. ``как сделать'' [``how to'' vs. ``what is'']} {\it см.} декларативное vs. императивное знание
\item {Чёрча--Тьюринга тезис} {\it см.} тезис  Черча--Тьюринга
\bigskip
\item {Шамир, Ади [Adi Shamir]} 67{\it п}
\item {шахматы, задача о восьми ферзях [chess, eight-queens puzzle]} 130 (упр.~2.42), 389 (упр.~4.44)
\item {\texttt{шишка} (правило)} {\it 413}, 428 (упр.~4.65)
\item {Шмидт, Эрик [Eric Schmidt]} 225{\it п}
\item {Шроуб, Ховард~Э. [Howard~E. Shrobe]} 406{\it п}
\bigskip
\item {Эдинбургский университет [University of Edinburgh]} 405{\it п}
\item {Эйлер, Леонард [Leonhard Euler]} 83 (упр.~1.38)
  \subitem {доказательство Малой теоремы Ферма} 65{\it п}
  \subitem {ускоритель рядов} 315
\item {Эйндховенский технический Университет} 294{\it п}
\item {экспоненциальный рост [exponential growth]} 58
\item {электрические цепи, моделируемые с помощью потоков} 322 (упр.~3.73), 328 (упр.~3.80)
\item {элементарное ограничение [primitive constraint]} 272
\item {элементарные выражения [primitive expressions]} 25
  \subitem {вычисление} 30
  \subitem {имя переменной} 28
  \subitem {имя элементарной процедуры} 26
  \subitem {числа} 26
\item {элементарные процедуры [primitives]}
  \subitem {\texttt{<}} 36
  \subitem {\texttt{>}} 36
  \subitem {{\tt *}} 26
  \subitem {{\tt +}} 26
  \subitem {{\tt -}} 26, 36{\it п}
  \subitem {{\tt /}} 26
  \subitem {{\tt =}} 36
  \subitem {{\tt apply}} 182{\it п}
  \subitem {{\tt atan}} 174{\it п}
  \subitem {{\tt car}} 95
  \subitem {{\tt cdr}} 95
  \subitem {{\tt cons}} 95
  \subitem {{\tt cos}} 81
  \subitem {{\tt display}} 96{\it п}
  \subitem {{\tt eq?}} 148
  \subitem {{\tt error} {\em (нс)}} 80{\it п}
  \subitem {{\tt eval} {\em (нс)}} 360{\it п}
  \subitem {{\tt list}} 108
  \subitem {{\tt log}} 82 (упр.~1.36)
  \subitem {{\tt max}} 103
  \subitem {{\tt min}} 103
  \subitem {{\tt newline}} 96{\it п}
  \subitem {{\tt not}} 37
  \subitem {{\tt null?}} 110
  \subitem {{\tt number}} 150
  \subitem {{\tt pair?}} 117
  \subitem {{\tt quotient}} 312 (упр.~3.58)
  \subitem {{\tt random} {\em (нс)}} 66, 221{\it п}
  \subitem {{\tt read}} 356{\it п}
  \subitem {{\tt remainder}} 60
  \subitem {{\tt round}} 198{\it п}
  \subitem {{\tt runtime} {\em (нс)}} 67 (упр.~1.22)
  \subitem {{\tt set-car!}} 242
  \subitem {{\tt set-cdr!}} 242
  \subitem {{\tt sin}} 81
  \subitem {{\tt symbol?}} 151
  \subitem {{\tt vector-ref}} 490
  \subitem {{\tt vector-set!}} 490
\item {элементарный запрос [primitive query]} {\it см.} простой запрос
\item {элементы вычисления первого класса [first-class elements of computation]} 87, {\it см. также} полноправные элементы вычисления
\item {Эратосфен} 308{\it п}
\item {эффективность [efficiency]} {\it см.~также} порядок роста
  \subitem {вычисления} 365
  \subitem {доступа к базе данных} 418{\it п}
  \subitem {древовидно-рекурсивного процесса} 57
  \subitem {компиляции} 518
  \subitem {Лиспа} 24
  \subitem {обработки запросов} 420
\item {Эшер, Мориц Корнелис [Maurits Cornelis Escher]} 132{\it п}
\bigskip
\item {явное кодирование элементарных процедур [open coding of primitives]} 547 (упр.~5.38), 550 (упр.~5.44)
\item {язык [language]} {\it см.} естественный язык; язык программирования
\item {язык запросов [query language]} 406, 407
  \subitem {vs. математическая логика} 425
  \subitem {абстракция} 412
  \subitem {база данных} 407
  \subitem {логический вывод} 415
  \subitem {правило} {\it см.} правило (в языке запросов)
  \subitem {проверка на равенство} 412{\it п}
  \subitem {простой запрос} {\it см.} простой запрос
  \subitem {расширения} 428 (упр.~4.66), 447 (упр.~4.75)
  \subitem {составной запрос} {\it см.} составной запрос
\item {язык описания изображений [picture language]} 133
\item {язык программирования [programming language]} 22
  \subitem {логического} 406
  \subitem {объектно-ориентированный} 195{\it п}
  \subitem {с аппликативным порядком вычислений [applicative-order]} 370
  \subitem {с нормальным порядком вычислений [normal-order]} 370
  \subitem {сверхвысокого уровня} 40{\it п}
  \subitem {сильно типизированный} 328{\it п}
  \subitem {строение} 369
  \subitem {функциональный} 333
\item {язык регистровых машин [register machine language]}
  \subitem {\texttt{assign}} 455, 471
  \subitem {\texttt{branch}} 454, 471
  \subitem {\texttt{const}} 455, 471
  \subitem {\texttt{goto}} 454, 471
  \subitem {\texttt{label}} 454
  \subitem {\texttt{op}} 455, 471
  \subitem {\texttt{perform}} 457, 471
  \subitem {\texttt{reg}} 455, 471
  \subitem {\texttt{restore}} 467, 471
  \subitem {\texttt{save}} 467, 471
  \subitem {\texttt{test}} 454, 471
  \subitem {\texttt{метка}} 454
  \subitem {команды} 454, 471
  \subitem {точки входа} 454
\item {язык сверхвысокого уровня [very high-level language]} 40{\it п}
\item {ячейка, в реализации сериализатора [cell, in serializer implementation]} 295
\end{theindex}
